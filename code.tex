\href{index.html}{home}

\section{Code}\label{code}

\subsection{Philosophy}

All our code of all kinds is organic media as described in previous chapters.

All code can be copied by all users. All code is human readable, and can be copied by copy/paste.  All code can be edited on all servers by all users.  All code can be deleted by all users on all servers.  No native code.  All code is run from a browser, and can be run from any browser on the local network.  All code is part of a server which can be replicated as a block with a single operating, copying all files from any computer on a local network to any other computer on that network.  Each instance of the code is totally self-contained, and can be edited and replicated to create a new node with no relations of any kind with any other node.  A single Raspberry Pi could be used to replicate a server to a laptop on a coffee shop which pushes the code to a public git repository which is then replicated to a globally visible url, which replicates to more computers, git repos, terminals, and servers, and so on, replicating a single instance to the whole world of billions of devices in a matter of minutes if the relations between network nodes are set up properly, all with no centralized structure of any kind.  

This system does not have ``users'' in the sense that they exist in current information technology systems.  The systems just \emph{are}.  Think of them like any other appliance outside of information technology.  We will illustrate the idea of a commonly shared thing with an example.

When you go into a coffee shop, the counter where orders come out and are picked up by customers does not have divisions by property.  Customers are not assigned a special area on the counter to the exclusion of others.  Likewise the baristas do not have zones marked off that are theirs, with some complex system for assigning a space over from barista to customer(like the driver and rider in a ride sharing app).  Instead, the counter is a shared resource: its value to customer and barista alike is maximized by it being a common space where cups of coffee appear and are taken based on whatever is most convenient at the moment.  Sometimes someone breaks the model here, and throws a huge dirty backpack on the counter or spills something sticky everywhere, putting the system out of use.  But again we all accept that the risk of this and the added work to correct it is outweighed by the benefit of the counter being an open space.  

Our information systems in Geometron are like the counter top in a coffee shop.  Indeed, one of the primary use cases is to have the software run on a server literally sitting on the counter of a coffee shop.  As with the counter, the benefit to customers, workers, and owners of the coffee shop is such that it simply has more value shared with no property than it would if there were \emph{any} barriers to the flow of information.  Anyone with any experience designing and building information technology systems will need to constantly remind themselves how different a model this is to any existing system.  Even anonymous message boards are not structured this way.  Users are still users, even if they're just an anonymized IP address.  In our model there \emph{simply are no users!!}  No users means no passwords. It also means there are no ``transactions'' involving money or property, as any application which actually does commerce can't function without a user on each side of the transaction.  Our media can \emph{enable} all kinds of commercial transactions! People can post things which say how to get ahold of them, or link to pages with more specific content some of which might be private.  But the point is that something like an advertisement for some good or service on a local network is worth more the more people see it.  If members of a community are all posting to each other and all getting value from seeing those posts, it is in everyone's best interest to have things be as free as possible, and this means no users.

Without users, there are also no passwords, no databases, and no ``security'' of any kind.  Security is physical in a physical space, and our networks are physically local and not on the open Internet.  In order for this to work, people need to not put any private information on the network, ever.  The only way users on their private devices interact with Geometron is through their web browser.  

All that said, encryption can be used in interesting ways in a free open physically local network node.  Files can be encrypted in a human-readable format, dropped on a server, and copied out across the web from server to server and laptop to laptop.  One can think of numerous applications for this.  In the initial version of Geometron that is being released with this book, there are no built in cryptography technologies of any kind.  But with the application development workflow described in this work users can easily build that tech in whatever direction they choose. Again, this is the power of a fully open system: you can sit in a coffee shop anonymously editing code, write a totally new kind of cryptography application, share with other users in the coffee shop, who then copy to github and share with the world, and in no time your application has spread to the whole world, untraceable to you, and with others forking the code, riffing on it and improving it.  Your project shared with one person on a coffee shop can come back to you in the form of a vast new ecosystem of derived code, shared freely across the global network and finally back to your coffee shop.  



\subsection{Structure}

All Geometron apps run in a web browser and so are composed of HTML, CSS and JavaScript.  PHP scripts are used to communicate between the web browser and the file system on the server(even if it is just a localhost server on a laptop).  

Each instance of Geometron has a file structure where various types of files are always stored.  Whatever directory on the web server a Geometron instance is in can have subdirectories which also have yet another Geometron instance. We aim to have the whole system be well under 10 megabytes, so a 1 gigabyte storage capacity server can host a hundred instances easily, each in its own subdirectory.  All Geometron applications are html files which sit in the root directory of that instance(although this might be a subdirectory on a server).  JavaScript libaries are stored in the directory 



The initial location of the Trash Robot/Geometron Thing code is at
\url{https://github.com/lafelabs/thing/}.

To create another instance of the full Trash Robot/Geometron system, we
copy a program called ``replicator.php'' into the main web directory of
the server. The raw code can be found at either locally on this server
at \url{php/replicator.txt} or globally on the original lafelabs Github
``thing'' repository at
\url{https://raw.githubusercontent.com/LafeLabs/thing/master/php/replicator.txt}.

We generally run Trash Robot/Geometron in one of three ways:

\begin{enumerate}
\def\labelenumi{\arabic{enumi}.}
\tightlist
\item
  Run it on a hosted remote server somewhere
\item
  Run it on a Raspberry Pi and serve it over a local wifi network.
\item
  Run it locally on a computer we are using for active development
\end{enumerate}

To host it on a remote server, we first buy a domain name representing a
local place which is not property: a public street, public park, public
body of water name for instance. We always choose obscure domains, do
not use .com, and avoid any personal information or names of businesses.
Then we pay for hosting service. We find the root directory for web
hosting, and create a new file called replicator.php. We copy the code
in the replicator into that and save it. Then we point a browser to
{[}your domain name{]}/replicator.php and wait for the script to copy
all the files.

To run it on a Raspberry Pi, after installing the normal Pi software,
install Apache and PHP as follows:

Then install the \href{https://github.com/lafelabs/thing/}{Geometron
software} type copy/paste these commands into the terminal:

To run on a local laptop as localserver, if you're on a mac, just open a
terminal. You can use the ``command'' button combined with searching for
``terminal'' to find it, then pin it to the menu bar. On a Windows
machine,
\href{https://ubuntu.com/tutorials/ubuntu-on-windows\#1-overview}{install
Ubuntu under windows}. Then as with mac you can use control-escape to
bring up the Start Menu, and type in ``ubuntu'' and click on it to open
a terminal. Once the terminal is open, pin Ubuntu to the task bar for
easy use in the future.

In the terminal, you want to type

Or open .bashrc

And copy this line after the last existing line of the file:

And then just hit the letter ``s'' every time you get to the command
line.

When the local PHP server is running you can open a browser on that
machine and point it to \href{http://localhost/}{http://localhost} and
you will be running the full Trash Robot/Geometron software on that
machine. You can use this for purely local interaction where no one in
the world can see what you do, and can edit various files which you then
paste into other instances of the software, send to other users, or
import when other users send you
date(\href{scrolls/scrolls.md}{scrolls}, \href{scrolls/maps.md}{maps},
\href{scrolls/feeds.md}{feeds}, \href{scrolls/geometron.md}{symbols}).

You can fork the whole software when you run it locally on a laptop by
replicating the whole system into a directory which is a Git repository,
then pushing the code to a public repository(like on Github) and then
replicating the new version of the code to the whole Web by pointing the
code in replicator.php which has a url for ``dna.txt'' to the global url
for your dna.txt file. Dna.txt has all the files to copy organized by
type. Replicator.txt uses that to figure out what to copy. The DNA is
generated using another PHP script called dnagenerator.php. PHP files
are all stored as .txt files in the directory php, and a script called
text2php.php copies all of those files to the main web directory and
changes the extensions from .txt to .php.

All code is edited with the program \url{editor.php}. This is a code
editor which edits all code directly on the server. This is how all code
development works in Trash Robot/Geometron. It is all in the Web
Browser. Code formatting is carried out using the free
\href{https://ace.c9.io/}{open JavaScript library Ace.js}, hosted on
Cloudflare CDN at
\url{https://cdnjs.cloudflare.com/ajax/libs/ace/1.2.6/ace.js}. With this
we can edit all the HTML, all the JavaScript, all the PHP, the raw
Geometron, and various data files. This editor is used to make and edit
all kinds of files.

To create a new file we can use ``newfile'' after editor.php as follows:
editor.php?newfile={[}filiename{]}. The file will appear at the very end
of the list of files, with the right color coding and syntax
highlighting based on the file extension.

A coffee shop-centered community code work flow is now described. A
Raspberry Pi sits on the coffee shop wifi network. All users in the shop
share in making scrolls, maps, symbols, feeds, pages and apps. Then any
user can back all that up to a full new code instance, and push that to
their public facing Github page. That copy of replicator.php is the
pointed to that copy of dna.txt. The next instance of the software can
use the code from this new replicator.php and it will clone the whole
code base of that coffee shop, with no reference at all to the original
code. Each fork creates a fully independent copy of the code.

To fork a whole full instance of the software down a level, use
\url{fork.html}. This lets you create new branches with whatever name
you want, as well as delete whole branches. Deletion is real!! There are
no backups. We prevent data loss with massive redundancy of replication.
If all users frequently not only replicate but pass along all
information, loss is a normal part of information life cycle and easy
deletion is healthy.
