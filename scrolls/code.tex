\href{index.html}{home}

\section{Code}\label{code}

There are five basic types of code used in the Geometron System,
corresponding to five Alchemy archetypes. These are:

\begin{itemize}
\tightlist
\item
  Air: CSS
\item
  Water: HTML
\item
  Fire:JavaScript
\item
  Earth:Geometron Hypercube
\item
  Aether:PHP
\end{itemize}

Geometron is \href{scrolls/geometron.md}{documented here}. The other
four can all be learned from
\href{https://www.w3schools.com/}{W3schools.com}, using
\href{https://developer.mozilla.org/en-US/}{Mozilla's documentation of
web developer technology} and \url{https://www.php.net/} as a reference
for PHP.

All code self-replicates.

All code is human readable.

All code can be edited by all users.

All code can be deleted by all users.

All code can be copied by all users.

The initial location of the Trash Robot/Geometron Thing code is at
\url{https://github.com/lafelabs/thing/}.

To create another instance of the full Trash Robot/Geometron system, we
copy a program called ``replicator.php'' into the main web directory of
the server. The raw code can be found at either locally on this server
at \url{php/replicator.txt} or globally on the original lafelabs Github
``thing'' repository at
\url{https://raw.githubusercontent.com/LafeLabs/thing/master/php/replicator.txt}.

We generally run Trash Robot/Geometron in one of three ways:

\begin{enumerate}
\def\labelenumi{\arabic{enumi}.}
\tightlist
\item
  Run it on a hosted remote server somewhere
\item
  Run it on a Raspberry Pi and serve it over a local wifi network.
\item
  Run it locally on a computer we are using for active development
\end{enumerate}

To host it on a remote server, we first buy a domain name representing a
local place which is not property: a public street, public park, public
body of water name for instance. We always choose obscure domains, do
not use .com, and avoid any personal information or names of businesses.
Then we pay for hosting service. We find the root directory for web
hosting, and create a new file called replicator.php. We copy the code
in the replicator into that and save it. Then we point a browser to
{[}your domain name{]}/replicator.php and wait for the script to copy
all the files.

To run it on a Raspberry Pi, after installing the normal Pi software,
install Apache and PHP as follows:

Then install the \href{https://github.com/lafelabs/thing/}{Geometron
software} type copy/paste these commands into the terminal:

To run on a local laptop as localserver, if you're on a mac, just open a
terminal. You can use the ``command'' button combined with searching for
``terminal'' to find it, then pin it to the menu bar. On a Windows
machine,
\href{https://ubuntu.com/tutorials/ubuntu-on-windows\#1-overview}{install
Ubuntu under windows}. Then as with mac you can use control-escape to
bring up the Start Menu, and type in ``ubuntu'' and click on it to open
a terminal. Once the terminal is open, pin Ubuntu to the task bar for
easy use in the future.

In the terminal, you want to type

Or open .bashrc

And copy this line after the last existing line of the file:

And then just hit the letter ``s'' every time you get to the command
line.

When the local PHP server is running you can open a browser on that
machine and point it to \href{http://localhost/}{http://localhost} and
you will be running the full Trash Robot/Geometron software on that
machine. You can use this for purely local interaction where no one in
the world can see what you do, and can edit various files which you then
paste into other instances of the software, send to other users, or
import when other users send you
date(\href{scrolls/scrolls.md}{scrolls}, \href{scrolls/maps.md}{maps},
\href{scrolls/feeds.md}{feeds}, \href{scrolls/geometron.md}{symbols}).

You can fork the whole software when you run it locally on a laptop by
replicating the whole system into a directory which is a Git repository,
then pushing the code to a public repository(like on Github) and then
replicating the new version of the code to the whole Web by pointing the
code in replicator.php which has a url for ``dna.txt'' to the global url
for your dna.txt file. Dna.txt has all the files to copy organized by
type. Replicator.txt uses that to figure out what to copy. The DNA is
generated using another PHP script called dnagenerator.php. PHP files
are all stored as .txt files in the directory php, and a script called
text2php.php copies all of those files to the main web directory and
changes the extensions from .txt to .php.

All code is edited with the program \url{editor.php}. This is a code
editor which edits all code directly on the server. This is how all code
development works in Trash Robot/Geometron. It is all in the Web
Browser. Code formatting is carried out using the free
\href{https://ace.c9.io/}{open JavaScript library Ace.js}, hosted on
Cloudflare CDN at
\url{https://cdnjs.cloudflare.com/ajax/libs/ace/1.2.6/ace.js}. With this
we can edit all the HTML, all the JavaScript, all the PHP, the raw
Geometron, and various data files. This editor is used to make and edit
all kinds of files.

To create a new file we can use ``newfile'' after editor.php as follows:
editor.php?newfile={[}filiename{]}. The file will appear at the very end
of the list of files, with the right color coding and syntax
highlighting based on the file extension.

A coffee shop-centered community code work flow is now described. A
Raspberry Pi sits on the coffee shop wifi network. All users in the shop
share in making scrolls, maps, symbols, feeds, pages and apps. Then any
user can back all that up to a full new code instance, and push that to
their public facing Github page. That copy of replicator.php is the
pointed to that copy of dna.txt. The next instance of the software can
use the code from this new replicator.php and it will clone the whole
code base of that coffee shop, with no reference at all to the original
code. Each fork creates a fully independent copy of the code.

To fork a whole full instance of the software down a level, use
\url{fork.html}. This lets you create new branches with whatever name
you want, as well as delete whole branches. Deletion is real!! There are
no backups. We prevent data loss with massive redundancy of replication.
If all users frequently not only replicate but pass along all
information, loss is a normal part of information life cycle and easy
deletion is healthy.
