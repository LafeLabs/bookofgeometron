\href{index.html}{home}

\href{maps/maps}{map of map tools}

\section{Maps}\label{maps}

Maps are a format in Trash Robot/Geometron which are a generalized meme.
They represent an ordered list of objects, each of which has a position
in a rectangular area on the screen. Each element in the ordered array
has an x and y position and width all normalized to the size of the
square area, as well as an angle in degrees. The other properties each
element has are a url for an image if they're an image, HTML text for
both if they are not an image and for alt text if they are, and a link
destination which can be either a url or a map or scroll link inside the
geometron system. Maps can link to scrolls as well as other maps. Also,
each element has a Boolean variable ``maplinkmode'' which is false if it
is just a normal HTML link and true if it is a map or scroll link. Maps
are all stored in the ``\url{maps/}'' sub-directory of each Trash
Robot/Geometron instance. They are in
\href{https://www.json.org/json-en.html}{JSON} format.

Scrolls are all stored in the \url{scrolls/} directory. Links inside the
Geometron system are identified as to whether they are scolls or maps by
the full name of the file. For instance one would link to this scroll
from anywhere in the system using the name ``scrolls/maps'' as the
destination of either a link in a map element which has maplinkmode set
to ``true'' or in a hyperlink in the markdown format of the
\href{scrolls/scrolls}{Scroll}.

Maps are defined with the JavaScript library ``mapfactory.js'' which is
in the ``jscode'' directory at \url{jscode/mapfactory.js}.

Maps are created in Javascript by for example in a DIV element called
``mainmap'' with following code:

Maps are edited using the program \url{mapeditor.html}. Click on all the
things at random to figure out how to use that program. Save often.
Copy/paste JSON code from the text area to share maps across the
Internet or privately with other users. You can email JSON code, store
it, copy it etc, and anyone can import it with a paste into their
Geometron instance and save it locally on their server. This generalized
meme format replaces both meme making software and PowerPoint as well as
a large number of HTML frameworks and formats. It allows for a
generalized system for encoding information on an image, which can be
critical to documenting self-replicating physical technology. The three
pillars of all Geometron/Trash Robot software are the Map, the Scroll,
and the Symbols which are created with the Geometron language. This
``symbol'' is generalized to include those made in all physical media,
so that includes things like lab-on-chip fluidic circuits, hybrid
upcycled electronic circuits, laser cut shapes etc. Once Geometron is
used to encode all human language and all symbols and also all
technology, it can drive the hardware which displays maps and scrolls.
When all of this lives on fully upcycled hardware, the system if fully
metabolized and we can build self-replicating technology that does not
have any mining, money, or property, the ultimate goal of Trash Magic.

\subsection{Deletion}\label{deletion}

Maps are deleted with \url{mapdelete.html}. Just click ``delete'' to
delete. Be careful, there is no backup. Also on public servers this
might break, as do all file creation and editing functions from time to
time. It will work instantly on a \href{scrolls/terminal}{Raspbery Pi
Terminal}.

\subsection{Replication}\label{replication}

When you create a new map, run \url{dnagenerator.php}, and the next time
the whole tree is replicated that map will come along for the ride. To
replicate a specific map, find the URL of that map and use copy.php. The
syntax is

The ``from'' url can be anywhere on the Open Web or anywhere visible on
the local network. For example,
\href{https://www.pastebin.com}{pastebin.com} or a raw code link on
\href{https://www.github.com}{Github}

\subsection{Map editor Icon Meanings}\label{map-editor-icon-meanings}

Save: \includegraphics{iconsymbols/save.svg} Increment current element:
\includegraphics{iconsymbols/upelement.svg} Increment current element:
\includegraphics{iconsymbols/downelement.svg} Move current element down:
\includegraphics{iconsymbols/elementdown.svg} Move current element down:
\includegraphics{iconsymbols/elementup.svg} Delete current element:
\includegraphics{iconsymbols/delete.svg} Create new element:
\includegraphics{iconsymbols/add.svg} Remove image from element:
\includegraphics{iconsymbols/deleteimage.svg} Remove link from element:
\includegraphics{iconsymbols/deletelink.svg}
