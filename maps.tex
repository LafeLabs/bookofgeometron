
\section{Maps}\label{maps}

Replication replication replication replication replication replication replication

replication replication replication replication replication replication

maps are designed for maximum replication of themselves, ease of sharing by text and copy paste. Ease of building apps to use and edit.  And they are created to maximize the Geometron user's ability to replicate other technology.  

figures:

map editor


Maps are a format in Trash Robot/Geometron which are a generalized meme.
They represent an ordered list of objects, each of which has a position
in a rectangular area on the screen. Each element in the ordered array
has an x and y position and width all normalized to the size of the
square area, as well as an angle in degrees. The other properties each
element has are a url for an image if they're an image, HTML text for
both if they are not an image and for alt text if they are, and a link
destination which can be either a url or a map or scroll link inside the
geometron system. Maps can link to scrolls as well as other maps. Also,
each element has a Boolean variable ``maplinkmode'' which is false if it
is just a normal HTML link and true if it is a map or scroll link. Maps
are all stored in the ``\url{maps/}'' sub-directory of each Trash
Robot/Geometron instance. They are in
\href{https://www.json.org/json-en.html}{JSON} format.

Scrolls are all stored in the \url{scrolls/} directory. Links inside the
Geometron system are identified as to whether they are scolls or maps by
the full name of the file. For instance one would link to this scroll
from anywhere in the system using the name ``scrolls/maps'' as the
destination of either a link in a map element which has maplinkmode set
to ``true'' or in a hyperlink in the markdown format of the
\href{scrolls/scrolls}{Scroll}.

Maps are defined with the JavaScript library ``mapfactory.js'' which is
in the ``jscode'' directory at \url{jscode/mapfactory.js}.

Maps are created in Javascript by for example in a DIV element called
``mainmap'' with following code:

Maps are edited using the program \url{mapeditor.html}. Click on all the
things at random to figure out how to use that program. Save often.
Copy/paste JSON code from the text area to share maps across the
Internet or privately with other users. You can email JSON code, store
it, copy it etc, and anyone can import it with a paste into their
Geometron instance and save it locally on their server. This generalized
meme format replaces both meme making software and PowerPoint as well as
a large number of HTML frameworks and formats. It allows for a
generalized system for encoding information on an image, which can be
critical to documenting self-replicating physical technology. The three
pillars of all Geometron/Trash Robot software are the Map, the Scroll,
and the Symbols which are created with the Geometron language. This
``symbol'' is generalized to include those made in all physical media,
so that includes things like lab-on-chip fluidic circuits, hybrid
upcycled electronic circuits, laser cut shapes etc. Once Geometron is
used to encode all human language and all symbols and also all
technology, it can drive the hardware which displays maps and scrolls.
When all of this lives on fully upcycled hardware, the system if fully
metabolized and we can build self-replicating technology that does not
have any mining, money, or property, the ultimate goal of Trash Magic.

\subsection{Deletion}\label{deletion}

Maps are deleted with \url{mapdelete.html}. Just click ``delete'' to
delete. Be careful, there is no backup. Also on public servers this
might break, as do all file creation and editing functions from time to
time. It will work instantly on a \href{scrolls/terminal}{Raspbery Pi
Terminal}.

\subsection{Replication}\label{replication}

When you create a new map, run \url{dnagenerator.php}, and the next time
the whole tree is replicated that map will come along for the ride. To
replicate a specific map, find the URL of that map and use copy.php. The
syntax is

The ``from'' url can be anywhere on the Open Web or anywhere visible on
the local network. For example,
\href{https://www.pastebin.com}{pastebin.com} or a raw code link on
\href{https://www.github.com}{Github}

\subsection{Map editor Icon Meanings}\label{map-editor-icon-meanings}

Go through in detail how to use the editor. Describe the specifications to build a better editor, plead with developers reading the book to write a better one.  

\subsection{Examples}

Use cases. 

annotated screen shot or image of geographic map.  Example of location places on a photo of a tourist map with a DC subway exit map. Example of a screen shot from openstreetmaps.org

Location of a physical object in a photograph of a place, with link to file, page, scroll or map
 
navigation: simple links to other documents on the local Geometron system as well as to Geometron apps like feeds or symbol programs, and also links to other web pages.

Linking from a global page to a local terminal and vice versa, with photographs of the terminal uploaded to the terminal.

memes. Just regular old memes, but with edit capability and the ability to share

graph theory diagrams using geometron symbols combined with text, use of math via MathJax js library.

more generalized replacement for PowerPoint or Keynote, but free and open and readily replicated.  Give example of replicating the whole deck of maps from one server to another using the standard code replication workflow, separate from replication of the whole system.  

labels on a physical object to document that object.  Labels can be links to further documentation of the object, which themselves have further zoomed in detail photographs of the object.  Detailed, hyperlinked fractal documentation of physical objects can really help replicate those objects, which is what our whole system is for.

Geometric memes showing how geometric symbols fit over objects in the environment, connecting physical things to geometric abstractions like the pentagon or hexagon.

\subsection{conclusions}

Summary of use: all our media is designed for free replication.  This means that it's easy to find a thing, easy to copy it, and easy to share it.  Maps help us to locate things in physical space to share by annotating geographic maps as well as photos of places.  They help us replicate technology by rapidly and freely creating documents of details of the object, with links to further documentation of finer and finer elements of the thing until all parts are sufficiently documented to enable replication.  By replicating the pitch deck functionality of PowerPoint we further facilitate replication by helping people to communicate stories behind what we do, helping to convince people why they should replicate.  Finally, the ability to make memes easily which can be edited and remixed we build a more dynamic social media based on memes than is possible with bitmapped graphics.  This enables open brands to become virulent, getting more and more people invested in seeing our projects succeed and again furthering replication.





