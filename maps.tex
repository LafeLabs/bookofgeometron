
Maps are a from of document in which a set of images, words, and links are arranged geometrically on the screen.  Just as the Geometron Scroll can be thought of as a replacement for Microsoft Word, the Geometron Map can be thought of as a replacement for Microsoft PowerPoint and Keynote.  But it is also a way to make and share memes, to annotate geographic maps, annotate photographs of objects and really do any kind of communication where the relative geometry of objects matters.

It is worth once again examining the consumer civilization's version of this in more detail to understand what we are trying to do differently here.  PowerPoint is the language used in the replication of things in today's world.  When a new company is born, the founders use a PowerPoint slide deck to sell that company to investors.  Every government applied science project starts with PowerPoint slides(often just a single one in the infamous ``quad chart'' format).  In many ways our whole civilization runs on PowerPoint today.  It is hard to imagine any project being funded in the world today, be it government, corporate, or non-profit, without a series of these simple graphic constructions of text and images and vector graphics arranged in a geometric order of some kind.  And of course this format is the basis of the meme, this bizarre new type of thing that spreads freely across the web, generally with copy and pasted bitmaps.  

We also need this format for replication of technology in our trash-based civilization as well as for all kinds of other communication.  However, we need it to fit in with the values and laws of Geometron, and that requires that we rewrite the whole thing from scratch. 

As with every Geometron document format, this format has to be something human readable using plain text so that each individual Map can be pasted into a text message, email, or pastebin for freely sharing across the Web without any intermediary.  We do this using the same language as the Feeds discussed in the previous section: JSON(JavaScript Object Notation).  A Map is an array of objects, each of which has a collection of properties.  The array is denoted by a pair of square brackets, and each element is separated by a comma.  Each element is inside a pair of twiddle brackets, and consists of pairs of names of properties and values of those properties.  Each element has a position, a size, an angle, a text value(optional), a link(optional), an image(optional), and information on whether it is a global link or a local link inside the Geometron system.  

You don't need to understand what all that means to use the system! The technical description is just there for reference.  Maps are edited and drawn using a JavaScript library called mapfactory.js, which is replicated with each instance of Geometron.  If you are a web developer feel free to go read the source code now to get an idea of how it works(there is not much to it).

By default, maps are displayed in a square area on your screen, and when you load the Geometron home screen that square will be either the left or top part of your screen depending on if your screen is wider than tall(landscape) or taller than wide(portrait).  To load a map, just look at the list of maps in the menu either to the right of the screen(for landscape) or in the popup menu you click to open with the button and look at the right side list and click on any map.  That will load it.  Now you can try clicking around on all the maps on your system, as well as navigating from map to map using internal links inside the maps(some have this some don't). 

Maps are used to amuse, to tell stories, to make points, to denote where things are, to point out where a part of an object is located, and to connect web pages to one another.  Maps are much more powerful than the PowerPoint slides they replace for several reasons. The ability to have both global and local links to other documents makes them fully integrated into a global, ever-evolving network in a way that makes them much more rich and complex.  They are, like all Geometron documents, not owned by anyone.  Each individual map is a free document, which can be replicated, edited, deleted, shared an infinite number of times instantly by anyone on the system.  This creates a richness of information which is impossible with a dead file format like PowerPoint.  

Maps are a great way to build social media around a physical place.  When community forms around a local server in a local place, the local media should have photographs of the objects in the environment.  Unlike a consumer network made up of ``users'', we have people in a community who share documents openly and freely.  Photographs of people in our community can go here, and we can build up a media pool that includes us all, but as a shared community of documents, not as a database of ``users''.  It is also an important way of denoting the exact physical location of things if we are to build up complex systems of industrial production from trash in our immediate environment.  

Maps are also another way to rapidly create web pages in the Geomtron system, as the main home page can be set up to point to them directly by changing one line of code in index.html, by replacing loadscroll() with loadmap() and the name of a map.  All maps are stored in the directory maps/ on each Geometron Server, and you can see all the maps by looking there in a browser, then click on them to see and copy the raw text of the map.  Just as we have a home scroll stored in scrolls/home, there is a home map stored in maps/home.  

Maps are edited with the map editor, which is at mapeditor.html on your local Geometron Server.  The Map Editor looks a little bit different on a portrait versus landscape screen.  In either case, the screen is divided into different areas which have links and buttons to do different things.  The main map window should look the same as when you are in the passive map reading mode: a square either in the left or top of the screen.  The element edit window contains icons indicating various actions, including select next/previous element, move element up/down in list, delete element, create new element, save map, delete image, delete link, and selectors to select which type of object you can select from to replace in the element you are editing using the textfeed described below.  It is a good exercise to just try playing with these, deleting all elements, then making a fresh one and playing with that.  Whatever element is selected is moved by dragging around on the map display screen, which can be done either with drag-click with a mouse or touch-drag on a touchscreen.  Elements are resized or rotated using the zoom/rotate box, which contains slider bars for both scale and rotate as well as buttons for both zoom and rotate which only appear in landscape mode.

There is another window which lists all the maps, and you can click on those to select which one is being edited.  That window also contains links to the home screen, a link to the specific map you are editing, and a link to the map delete program.  The Map destroyer is exactly the same as the Scroll destroyer.  It is a list of all maps on the Server, with a button to delete each map instantly.  There is no undo.  To click is to destroy.  If a map is worth saving it is worth copying and sharing and if it is copied, deleting it costs nothing, so we make it very easy to delete.  The destroyer page has a link back to the map editor. 

The window with these links also has a text input where you can input the name of a new map.  Try entering the name of a new map, and you will see a blank screen.    To add an element to that map, click the icon with the plus sign.  To delete it click the red X.  Add another one, and move it to the top then bottom, move them around.  To add an image to a map element, click on one of the images in the window with the images. Then you can click the icon with the red X through the image symbol(mountains and a sun icon) to remove the image and go back to just text.  

That scroll of images can also display a scroll of links or text.  All of these are taken from a combination of feeds, but primarily the Text Feed, at textfeed.html. This is linked via an icon with squares separated by a triangle.  The local image feed is also displayed and you can add to that using links on here as well with ``choose file'' and ``upload image''.  The blue link icon will make that scroll a list of links, and the ABC icon will make it a list of text elements.  You can also edit all these manually using the table of inputs.  

Unlike the Scrolls, maps are not instantly updated as you edit.  They have to be saved with the save button. Whenever you save a map, the text based representation of the map in the JSON format is placed in the text area below the control buttons.  If someone sends you a link to a raw map, you can go copy the contents of that map to the clipboard of whatever device you are using and then paste it into that text area and click the ``import'' button to import it into the current map.  You can then save it, and the current map will be replaced by the imported map, destroying the existing map.  You can also hit the ``reset'' button to clear out the map and start over with a fresh one.   Try making a new map, then selecting all the text in the text area, and pasting it to a public pastebin, then sharing that link with another user.  They can then paste it into the text area of their server to make a new map which is a copy of your map, edit it, and send it along to the next person and so on.

 The button at the bottom of the table of inputs to edit the current map element after maplinkode sets whether that mode is true or false. If it is false, and there is a link from the map element it will be a regular hyperlink like any other link on the Web.  But if it is set to true, it can point to either any map or any scroll on the web.  If you paste a map in a pastebin and then get the link to the raw version of that pastebin, then put that into the ``link'' field in the table, you can click on that link and it will load that remote map from anywhere on the Web.  This can be incredibly powerful, and can create entire networks of complex interconnected maps, all via anonymous pastebins, all referencing other images around the Web without storing any information on the local server or linking to any specific server or user(there are no users).

The button marked ``height mode'' changes the relative height of the element rather than the height and width together, which can be useful for sizing text elements exactly how we want.  This does not do anything when the element has an image, however as the element will then automatically size around the aspect ratio of the image.  

While you can enter the address of an image, the value of a text area, or a link value into the text fields, this is unwieldy and particularly annoying on mobile.  Therefore the main way to insert one of these types of information into a map element is via clicking on it in the scroll of images, text or link.  This is switched between these three by clicking the appropriate icon in the control button table(there is a link, text, and image icon).   The image scroll is listing images from the local and global image feeds discussed in the last chapter, but it also has a special feed just for feeding information into the Maps.  This is the Text Feed, located at textfeed.html.  If you go to this page, you can input text, links, and images, and then go back to the map editor and use what you entered there.  This is important for working on mobile devices where the manual input in the table is very awkward, but it is also important for another reason: dealing with symbols, icons, and other creations of the Geometron geometric programming language.

Creating symbols with Geoemtron will take up much of the rest of this book, but while it works with vector graphics, we need a fast and simple way to embed them in maps.  To do this we use the so-called ``base 64 encoding'', which allows bitmap images to be inserted into files without reference to an external image file.   Symbols made with Geometron can be converted into base 64 encoding using the link from the textfeed page which says ``png code'', and that will let you choose a symbol to convert and save into the textfeed, along with the ability to select how large that image should be in pixels.  As with all Feeds in Geometron, you can delete any element of any of the sub-feeds at any time by clicking the big red X.  Textfeed also links to some other applications which use it, such as the Duality and Poetry Engine apps, which you can explore as well.

Just as with Scrolls, we need to have the ability to use mathematical typesetting with maps.  While most people may not need to typeset math, it is useful to know how to point mathematically minded people to these tools so that they can use them if they want.   The starting point for this is to find a map or scroll that references math, and they should link to the relevant pages.  The relevant pages are mathmapeditor.html, mathmapeditor.php, and mathuser.php.  Combining math typesetting with geometric symbols quickly and freely and sharing them can build a whole new method for rapid communication of mathematical ideas.  Setting up this system in physical proximity to a location with a community of mathematicians can be very powerful and is highly recommended!  

Now that you know how to make, edit, delete, and share Geometron Maps I want to say a few more words about exactly how they can be used in helping with the workflow of the overall system.  One of the most fundamental tasks we need to do to build a physical network on the Street is to help people to find things in the physical world quickly.  Global map servers exist with maps of the whole world, which everyone can get to on any Internet enabled device.  We can use these maps to find where we are, then screen shot the map, save it, upload it to a server, and build Geometron Maps which then annotate that map with symbols, links, and words to show exactly where a Server is, or where physical resources are for our trash-based industrial production.  This can also be very useful for ad hoc mapping in an area with specific local maps, such as near the exit of a subway.  Subway station exits often have a very specific localized map centered on the station with clear markings of relevant landmarks. If you photograph that, upload that photograph to a public image server, link to it, and annotate it, you can build a whole system of networked maps from that.  Don't forget links!  Maps can have annotations which are maplinks, linking to other maps or to scrolls with any kind of document you might want to associate with a place.  A physical place can have a rich fractal structure of documents built around it, forming a kind of physical hypertext: a document in which we and everyone around us are physically immersed.

When we want to present a graphical story like the ``pitch decks'' which use PowerPoint in the consumer media, we can treat each Map as a slide, and then link them together with links on the edge of the screen going from previous to next and so on.  This can copy the functionality of slide decks exactly, but with much more richness of content since they can also link to scrolls, do not have to be linear, and are not restricted to loading local maps.  Rather than a ``deck'' being a dead document sitting on a private hard drive, linked Map sets of Geometron join a global swarm of potentially trillions of linked documents all replicating, evolving, and being destroyed in an ever shifting living informational universe.

When we build technology, and attempt to document that technology to aid in replication, we need to be able to label parts of the technology, and then link to images of sub-systems which then also have annotations.  Maps allow us to do this in an infinitely fractal way, zooming in on things with more and more detail, and then forking off to related things which are linked in much the way that Wikipedia forms a vast fractal network of information.  

The combination of geometric programming discussed in the next sections with Map creation allows for exploration of symmetries of the world around us which is unlike any other tool in its direct connection to the symmetries and scales of the Universe.  This will be explored much more as we go along, but I encourage the reader to keep returning to the Map editor and making maps with all the other parts as we go along and learn the Geometron System.
 
Beyond these examples, there really are no limits to how much can be done with this format.  It is lightweight, simple, easy to copy, and fundamentally more powerful than the predatory consumer systems like PowerPoint.  The framework is so simple that interested programmers can easily rewrite the whole thing from scratch in many systems, making all sorts of applications with it.  As with the JSON format itself, or HTML, the very simplicity of the specification is what gives it its power.  A blank PowerPoint document with literally no content can still be well over a megabyte, and you can't read it or edit it without paying Microsoft rent for the ``right'' to run code already on your hard drive which they agree not to break if you keep paying them protection money.  A Geometron Map can be as small as a few bytes, is human readable, can be shared by text message or email with direct copy and paste, can replicate itself freely across the network, and can be opened by an application that is itself only a few kilobytes in size.  Build on this system! Make it yours! Share it and help it grow!




