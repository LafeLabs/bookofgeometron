\href{index.html}{home}

\section{Feeds}\label{feeds}

A Feed is a sequence of elements. The elements don't have geometric
structure like a Map. They can be text, links, symbols, or any other
kind of media. They are generally stored in the ``\href{data/}{data}''
directory as JSON format files which end with ``.txt'' so that they can
be read by humans in a browser.

The Feed is a general framework for building formats, but in the basic
Trash Robot server we implement a few versions.

\subsection{\texorpdfstring{\href{globalimagefeed.html}{Global Image
Feed}}{Global Image Feed}}\label{global-image-feed}

This is an array of image urls. This is a key component of how Icon
Tokens are made. We often start by doing an image search on the Web for
some symbol, logo, image, or icon. We then right click the image and
``copy image location'' to the clipboard. Then we drop the url in the
input in the global image feed to add it to the feed. Click the red
``x'' to delete the image. Image feeds can be exported from the text
area, copied, and pasted into the same window of any other Trash Robot,
imported and used anywhere on the Network. Since this data is just text
it can be sent via text message or email so that feeds can be privately
shared. The local image feed is stored at \url{data/imagefeed.txt}

We can make global image links in this Feed by uploading images to
\href{https://imgur.com/}{www.imgur.com}, then right clicking the image
to get the url and putting that url in the image feed. This method is
used to document much of the Trash Robot system or for general rapid
information sharing.

\subsection{\texorpdfstring{\href{linkfeed.html}{Link
Feed}}{Link Feed}}\label{link-feed}

This is a feed of ``links'' in a general sense which can be images,
links, or just text. They are edited using the ``operator screen'',
which should be in the link feed itself, and can be found at
\url{linkfeededitor.html}. Each element has three fields: ``href'',
``src'', and ``text'', which are the url the link points to, the image
if there is one, and the text. The data are stored on each Trash Robot
at \url{data/linkfeed.txt}. As with the image feed, the whole feed can
be copied, pasted, imported and exported using a text area, but in this
case it is on the editor screen not the feed display. The input is used
to put in urls of other link feed files. These can be anywhere on the
Web. This can be used to make anonymous pastebin links which are link
feeds which can display on any local Trash Robot without ever posting to
a global server, for private exchange of link feeds. f \#\#
\href{textfeed.html}{Text Feed}

The Text Feed is used for a number of Trash Robot applications. In spite
of its name, it is not just a feed of text, but consists of three
feeds(arrays): ``text'', ``src'' and ``href''. These really are what
they sound like, three feeds in one. Users can add links, add images,
add text, or delete any of them, and can copy and paste and share and
import feeds. Text feed has a number of functions in the Trash
Robot/Geometron system. It is used for the Map Editor as a source of
links, images, and text which do not need to be entered in a keyboard.
It is also used in the \href{poetryengine.html}{Poetry Engine} and
\href{duality.html}{Duality}. These are documented with the
\href{scrolls/poetryengine}{poetry engine scroll} and
\href{scrolls/duality}{duality scroll}.

\subsection{\texorpdfstring{\href{chaosfeed.html}{Chaos
Feed}}{Chaos Feed}}\label{chaos-feed}

Chaos Feed is a user friendly text feed. Type in the input to post. Hit
red ``x'' to delete. Nuke the feed with the explode emoji. Reload with
the arrow loop emoji. HTML works, so you can manually enter html for
links and images, allowing a link out to be added. Chaos Feed can be set
to be the top level of a Trash Robot Server for text feed sharing mayhem
and fun. Chaos feeds are stored at \url{data/chaosfeed.txt}.

\subsection{\texorpdfstring{\href{iconfeed.html}{Icon
Feed}}{Icon Feed}}\label{icon-feed}

This is a critical feed for the overall system work flow, as it is how
we share the Token Icons which are printed into clay. See the
\href{maps/workflow}{workflow map} for links to the elements of the
process by which these are made. Here again is where the copying,
pasting, importing and exporting of feeds is very important. Users can
create a whole feed of icons locally on a private server, then send that
via private message to other users anywhere in the world, who can then
edit on their own private servers, without any data ever leaking to the
public Internet, while still having no users and no databases on each
individual server.

\subsection{\texorpdfstring{\href{symbolfeed.html}{Symbol
Feed}}{Symbol Feed}}\label{symbol-feed}

This is not really a feed in the strict sense above, but it behaves like
a feed in the user interface. Every time a symbol is saved using
\url{symbol.html} an SVG and PNG file are both created, and these are
saved in a directory called \url{symbolfeed/}. These can be saved
locally and then used for anything. The pairs of files are also used
when programming the Dremel laser cutter to directly create laser cut
acrylic geometry shapes. The SVG files alone, with different layers as
different colors are used for the cut and etch layers when making laser
cut shapes ordered from \href{https://www.ponoko.com/}{Ponoko.com}.
Clicking on an SVG file also loads it up into \url{symbol.html},
including the structural JSON information which sets styles and
positions of the symbol.

\subsection{\texorpdfstring{\href{wall.html}{Wall}}{Wall}}\label{wall}

The Wall is a feed of one element. It is just a text document, stored at
\url{data/wall.txt}, which is edited and read by users. Type to edit.
Delete to delete. There are no users, no databases and no logins. Just
information freely shared.
