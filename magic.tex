
Magic is self-replicating desire.  When we desire a thing and that thing comes to pass, that is a form of replication from our minds into either the physical world or into the minds of others.  In practice, given the networked nature of modern society, it is the replication of desires into the minds of others which is overwhelmingly the more powerful today.  Our desire in this work is to create a new civilization built entirely from the waste streams of the old dying consumer society.  We also desire that this new world be one where people are free to carry out their desires to help one another, to build interesting art, to live in comfort and style, and to have adventures.  Only by firmly fixing in our minds this end desire and building a new system of knowledge from scratch with these ends in mind, can we break free of the existing system which forces us all to do work we hate to buy things we don't want which destroy the Earth and hurt other people. 

In order to build up this new way of seeing the world, we dive down to the deepest level of human thought, which is the subject of ``things'' in the most general sense.  What is a ``thing''? What is a collection of ``things''?  These are subjects mathematicians have tackled with vigor for a long time.  At the end of the 1800s and for the first half of the 20th century,  a collection of philosophers and mathematicians ...

magic: the replication of desire.  I want a thing.  I want to replicate a thing, and I want to replicate the replication of the desire to replicate the thing.  When many people desire a thing, anything is possible.  

trash magic: use of trash to replicate what we desire, and to replicate the desire to replicate that thing.

set magic, which replaces set theory.  An ontology built entirely on replication.  Replace the set of ZFC with a less format set the purpose of which is replication.

Icon Magic, Sigil Magic, Symbol Magic.  Use of the icon game pieces on sigil boards made with AG to play out various informational relations between concepts, move them around, manipulate them in order to understand them and communicate them to other people. Alchemy, the sequence of bags.

Brands as magic, the open brand, trash robot.  

alchemy.  language as a tool.  We can simply choose words and symbols and use them to achieve our ends.  This means using any available archetypes. 

