

\subsection{Consumer Civilization}

The nature of our present civilization is one of constant consumption.
We mine or drill material out of the ground, re-shape it or burn it for
fuel, and ultimately turn it into toxic waste which we dump back into
the environment. During the process from mine to landfill, we exchange
``money'' and this is what we call the ``economy''. The things which can
be exchanged for money are called ``property''.

Mining.  Each mineral is absolutely needed by whole system. Each is critical.  Rare earths.  Exotic transition metals. Lithium batteries.  foreshadowing of where all this is going by closing the loops, examples. 

Centralized mass production.  The fab.  

Globalized consumer supply web.

Universal replication of structure of waste stream.

The purpose of media in consumer civilization: to stimulate consumption.  Not just through advertising but through legitimizing the power structures required to maintain consumption, both corporate power and the power of the state.

\subsection{Organic Civilizations}

In mature civilizations, technology is all self-replicating. We look to the technological civilizations of
indigenous people as an example of what mature technology should look
like. These civilizations are based not on a stream from
mine to landfill, but on closed loops of material in equilibrium with
living ecosystems. 


An example of technology in a mature civilization is traditional wooden
boats. If people live in equilibrium with a forest, as old trees die or
are harvested new trees grow. People carve logs into boats, and then use
them to hunt and sustain their communities. A new generation of humans
is born from the old, and they are taught all the skills to copy the
construction of the boat, the stewardship and harvesting of the trees,
and how to pass along the knowledge of boat building(as well as the
hunting and fishing that sustains the whole society) to the next
generation after them and so on. Unlike a consumer civilization, this
type of self-replicating organic technology can continue in theory
indefinitely. If it has the resilience built in to improve and change
over time in response to changes in the ecosystem it can truly sustain
itself without limits. Mature civilizations like this have existed for
thousands of years all over the Earth.

Three things to note about a technology such as the one described above:
everything replicates, everything can be modified, and everything dies.
Every wood boat will rot, and eventually become soil again in the forest
which will produce more trees. The technology is soft enough that people
can modify it, carve and repair it over time, change it as needed, and
evolve it as needed in response to changing conditions in the
environment or new innovations in the technology. And of course, every
boat can always be copied freely. In this context the idea of property
doesn't make sense. Nothing is permanent. Everything is constantly going
through a cycle from soil to tree to boat to soil again, growing,
evolving dying, and being reborn.


\subsection{What comes next? Opportunities!}


The false dichotomy of ``progress''.  There will never be another stone age.

The structure of the trash feed.

A picture of what civ could look like based on trash feed.

\subsection{Goals and Methods}

is this its own section? do we just cut all this and have previous section point to organic media as the path forward?

So what is the alternative? The alternative is a switch from an
arithmetic to a geometric economy. This means that rather than an
economy of exchange, where numbers are used to represent both property
and labor, we have an economy of replication of geometric constructions.
In this situation, the replication of information \emph{is} the economy.
It does not turn into numbers at any point. One can create a thing via a
geometric construction from trash which contains information for its own
replication. This then replicates out into the human network in a
replication-based future descendant of the current Internet, and we all
share the increased value. When we create value we do not ``trade'' it
for something, because the amount of value will \emph{increase} as it
replicates, making any trade that truly represents the added value
impossible. In a post-scarcity trash-sourced geometric economy, everyone
benefits from the creation of creators because the best things naturally
replicate to all of humanity, including the back to the creators
themselves, but now amplified by the added benefit of billions of people
potentially improving it.

The Geometron/Trash Robot system is an attempt to build an information
technology system which serves this purpose: to build a purely geometric
economy based on replication. In this system, everything replicates,
everything evolves, and everything can be deleted at any time. In the
beginning we source some parts from trash and some parts from common off
the shelf consumer items that are easy to buy without any centralized
company. In order to replicate successfully, however, this information
networking system has to provide immediate value to the user, both
within the existing money-based economy, and also outside that system.

In order for all this to work, we have to give up the three main
elements of our existing system: mining, money, and property. As long as
money exists, replication will break the economy. As long as mining
exists, we will continue on a path to total destruction of the world and
eventually scarcity and extinction. And as long as property exists,
replication which could end all scarcity for all of humanity will be
hindered by the people who control everything. The extension of property
into the domain of pure information is another major reason that people
in the replication industries have been able to so brutally exploit the
rest of humanity. They claim to ``own'' most of the freely replicating
information we rely on in the new economy, and they're effectively
landlords who can create new land as many times as they want for free,
but then can charge rent on all of it. Only by creating a new economic
model from scratch which simultaneously abandons money, mining and
property, can we build a just and sustainable future.

It is not immediately obvious that geometry can completely replace
numbers-based thinking. But in most cases, our technology is already
completely driven by geometry, we just choose not to look at it that
way. What is a microchip? It is a geometric design imprinted in silicon.
That design is made by a geometric program interacted with by a user
thinking geometrically. Even the code itself, encoding into the physical
circuits, is a geometric pattern of either magnetic or electric
perturbations of physical objects. And ``computers'' are not used
primarily to compute things, but to display graphical information, both
text and pictures and graphics. The repeated actions of automation
machinery represent geometric operations(e.e. move over, move up, move
down, move over, etc\ldots{})

Saying ``geometric economy'' sounds abstract and to say it
self-replicates sounds far fetched, but none of this is a new idea. Many
of the oldest and most sustainable civilizations in the world have very
very old technologies involving textiles made from natural materials
woven into patterns. This is self-replicating geometry. It satisfies all
the properties listed here: it dies of natural decay, is taught as a
skill from the old to the young and is constantly replaced, and it can
be modified and improved over time by the constantly-replicating group
of experts in the technology within the community. So the transition to
a geometric economy is not to some futuristic new idea but a return to
ideas which predate the rise of industrial societies and nation-states.

All that said, we live in times different from any which came before.
The Internet is the baseline information system now for all of humanity.
Even the few people not connected to the Internet now have their lives
completely molded by it, as all power networks exist on it now. And
consumer society has injected every single element or type of finished
product our civilization uses all over the planet. So while in times
before now one group of people might discover the use of bronze and
another of steel, now every single person on the planet can get aluminum
sheet metal, titanium reinforced steel, rare earth metals, ultra high
purity silicon, etc. All these incredible materials, already shaped and
processed into the most useful form, are simply sitting in piles of
trash in every corner of every nation on the planet now, all directly
adjacent to nodes on an information network which connects to every
other part of the planet.

This combination of universal networking and universal access to
identical, standardized, trash elements creates a formula for building a
new information system from scratch from which a whole stream of new
civilizations can rise. It is our task to build a seed from which such
civilizations can be built. Geometron is an attempt to build this seed.

This chapter encapsulates the Trash Magic Manifesto and references it.  It sets up the problem at the highest level before delving into the specifics of the solution.  This chapter describes the problem that all subsequent chapters provide the solution for.  Essentially the Trash Magic Manifesto laid out a goal for technology and for civilization.  I realized that information technology has to form the basis of all the other technology.  If information technology is taken to be general enough, it can be the basis of the entire complete set defined in the Trash Magic Manifesto.  Define that set here.  Generalized media. Point to the next chapter.