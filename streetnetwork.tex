\section{Trash Robot/ Geometron Street
Network}\label{trash-robot-geometron-street-network}


Street Network:

\begin{itemize}
  \tightlist
  \item
  operators  
  \item
  terminals
  \item
  domains
  \item
  streets
  \item
  places
  \item
  developers
  \item
  signs
  \item 
  postcards
  \item
  markets
  \item
  feeds
  \item
  scrolls
  \item
  maps
  \item
  pages
  \item  
\end{itemize}


Me, now, here. \url{https://www.maplelawn.net}


The \href{scrolls/terminal.md}{Terminal} is a
\href{https://www.raspberrypi.org/}{Raspberry Pi} with a keyboard,
mouse, display and power supply, which run a web server only visible
over the local wifi network. It is carried by the Operator, who uses it
to help users create, edit, copy, and share files over the local
network.

The files on the Network can be: - \href{scrolls/scrolls.md}{Scrolls}.
These are a type of text file which uses the
\href{https://daringfireball.net/projects/markdown/}{Markdown markup
language} for formatting.\\
- \href{scrolls/feeds.md}{Feeds}. Feeds are either a directory with a
sequence of files a user can scroll through and select or an array of
any kind of information, be it images, symbols, words, links, etc. -
\href{scrolls/maps.md}{Maps}. A map is a sort of generalized meme, like
a PowerPoint or Keynote slide. It is an array of elements each of which
has a position, angle, width, possibly an image url, some text, and
possibly a link destination, which might be either a HTML hyperlink or
an internal link to a file on the system

All users on the same wifi network as the Terminal can view, edit,
delete, and copy all the files on the system. There are no user names,
no logins, no passwords, no private data, and no databases.

Users can all see all files, edit them, delete them, copy/paste them,
create new ones

Operators carry the Terminal around, share its link with people, talk to
people about the system, teach users to use the system, help to share
with new Operators.

Roles of the Operator:

\begin{itemize}
\tightlist
\item
  the keeper of the physical Terminal
\item
  maintain relationships with users of the Terminal and Domain
\item
  update the Geometron server at the hosted Domain with links to the IP
  address on the local wifi network of the Terminal, as well as the wifi
  network name and password or link to where to get it(e.g.~coffee shop
  register)
\item
  post ads for money or barter on the Geometron server at the hosted
  Domain for people who ask
\item
  post on the global Geometron server when and where the Operator will
  appear, or where the terminal is set up on what network if it's
  installed permanently.
\item
  Teach anyone who wants to learn how to be an Operator, recruit new
  Operators
\item
  Tell new users about the Network, teach them to use it, how to post,
  edit, delete
\item
  Promote any kind of business or other venture or project anyone
  physically local to the wifi network area has in exchange for barter
  with that user for useful things on location(including just a place to
  operate)
\item
  Spread Network into new places by finding a location with a wifi
  network, buying a domain and setting up hosting or getting someone to
  do that and pay for it,
\item
  Domain names spread in physical space using physical media with
  depiction of domain which points back to terminal ip address, wifi
  address and password, photo of terminal and operator other physical
  media(post cards, book marks, spray paint stencils), spreading the
  physical media with the domain name
\end{itemize}

Skills of Operator

\subsection{Domains}\label{domains}

\subsection{Terminals}\label{terminals}

\subsection{Laptops}\label{laptops}

get ubuntu working under windows, install apache and php

localhost

code goes from terminal to laptop to github to
