
\section{Street Network}



outline:
\begin{itemize}
  \item
  what and Why?  The power of the physical, local, and free.  Organic media, what we want, links to previous chapter.  Universal social media for sharing of information in a physically local domain with both content creation and consumption on all Web-enabled devices(laptops, phones, tablets, etc.)  Hybrid markets: like Craigslist, but way more local.  General description of what the system does(scrolls,maps, feeds, symbols, apps, industrial design and production via Trash Robot). This points to the subsequent chapters about these actual things.  Free boxes and food not bombs.  We are a hybrid between free boxes and food not bombs and craigslist.
  \item
  The Terminal.  This is the heart of this system. What is the Raspberry Pi, why is it powerful?  How to build the Terminal. Options involving big screens and projectors, public terminals with large publicly viewable displays.  How to adapt it to different situations, how to work with wifi networks, IP addresses, local and global, opening up a local wifi to a public domain.  how to avoid ANY property.  How the terminal is passed from user to user to operator to operator.  Data hygiene: how to keep all personal information of any kind of the machine, to prevent leaks of property.  What it means to have a network without property.  Replication paths via local laptops(localhost) on wifi, global github repos, replication to global hosted domains. How to constantly back up and replicate to avoid information death.  How to kill bad information.  How to nuke the whole system if it's too rotten.  Grey market and black market commerce.
  \item
  The Operators, what we are, what we do, how we do it, how we make money and barter, how we train new Operators .  Role of Operator as universal moderator.  Forking and avoiding the trap of the network monopolist.  How to transition between money to barter, how to scale up to an all barter system by providing value for 
  \item
  Psychogeography.  Nodes of power, examples at global level and local level, finding the nodal points.  Go through the whole philosophy, places, examples such as: parks, intersections, neighborhoods, famous landmarks, bridges, forking below a place, discussion of distance scales and granularity.  Targeting nodes of power: Sand Hill Road. Wall Street. SoMa.  K Street DC. Jackson Hole. Martha's Vinyard.  Navy Memorial DC. Use of power nodes to build extremely powerful local networks, where information is exchanged between players in existing power networks.  A network in the right coffee shop could connect people who collectively are processing 10's of billions of dollars of commerce just in that one coffee shop!
  \item
  domains.  choosing domains, buying them, sharing them, avoiding troubles. Entropy of domain choice, size of name space. use of free web hosting services.  Signs, markers, stencils, postcards: physical media which points to domains which point to terminals and link to the IP addresses.  The work flow with physical media to software media and back.
  \item
  The market
  \item
  the coffee shop(or pub) node.  How it can build community in the coffee shop, help all coffee shop customers to share and prosper, help neighbors of coffee shop, the developer workflow, use to expand operators.  Coffee shop network as service for coffee shop owner for barter for coffee and food at shop.  Building collaboration with businesses, both local and global, how to use global chains to scale globally with mutual aid and benefit for all.
  \item
  scaling up: the global swarm, going to full stack geometron(see last chapter), building more 
\end{itemize}

\subsection{What is the Street Network?}
\subsection{Building the Geometron Terminal}

Buy the stuff:

\begin{itemize}
\item
Raspberry Pi 4 board from Sunfounder
\item
SD card
\item
SD card reader
\item
Mini USB keyboard(without number pad)
\item
mouse
\item
Sunfounder HDMI display with 12 volt power supply and USB power out to drive PI, wall plug and HDMI cable
\item
12 V LiPo battery pack with wall plug charger from TalentCell(sold via amazon)
\item
wifi hotspot
\end{itemize}

Put it together.  Just assemble the Sunfounder terminal as per the instructions.  

Burn the card with NOOBS, put it in the Pi.

Use paint pens to put symbols on keyboard.

plug in mouse and keyboard.

Make a bag to carry the terminal around in, or find an appropriate backpack and sew symbol onto it.  Symbol of Raspberry Pi using Penrose Tiles.

Boot up the pi, set it up with no password

Install Apache and php

copy the Geometron code replicator script replicator.php into the web directory. 

Learn to use with subsequent chapters of this book, customize and deploy, replicate to other peoople

\subsection{The Operators}
\subsection{Psychogeography}
\subsection{Domains}
\subsection{Street Market}
We help people sell stuff directly on the Street, out in the open, with a sign advertising the Market.  People can sell for barter.  We also sell directly the items from Trash Robot and other Geometron Things described below.  These include the ArtBox as a purse, shirts, pants, flags, bags, clay icon tokens, robots, terminals, laser cut acrylic shapes and rulers and protractors, Pyramids,
\subsection{Coffee Shops and Pubs}
\subsection{Scaling Up}


Street Network:

\begin{itemize}
  \tightlist
  \item
  operators  
  \item
  terminals
  \item
  domains
  \item
  streets
  \item
  places
  \item
  developers
  \item
  signs
  \item 
  postcards
  \item
  markets
  \item
  feeds
  \item
  scrolls
  \item
  maps
  \item
  pages
  \item  
\end{itemize}






The \href{scrolls/terminal.md}{Terminal} is a
\href{https://www.raspberrypi.org/}{Raspberry Pi} with a keyboard,
mouse, display and power supply, which run a web server only visible
over the local wifi network. It is carried by the Operator, who uses it
to help users create, edit, copy, and share files over the local
network.

The files on the Network can be: - \href{scrolls/scrolls.md}{Scrolls}.
These are a type of text file which uses the
\href{https://daringfireball.net/projects/markdown/}{Markdown markup
language} for formatting.\\
- \href{scrolls/feeds.md}{Feeds}. Feeds are either a directory with a
sequence of files a user can scroll through and select or an array of
any kind of information, be it images, symbols, words, links, etc. -
\href{scrolls/maps.md}{Maps}. A map is a sort of generalized meme, like
a PowerPoint or Keynote slide. It is an array of elements each of which
has a position, angle, width, possibly an image url, some text, and
possibly a link destination, which might be either a HTML hyperlink or
an internal link to a file on the system

All users on the same wifi network as the Terminal can view, edit,
delete, and copy all the files on the system. There are no user names,
no logins, no passwords, no private data, and no databases.

Users can all see all files, edit them, delete them, copy/paste them,
create new ones

Operators carry the Terminal around, share its link with people, talk to
people about the system, teach users to use the system, help to share
with new Operators.

Roles of the Operator:

\begin{itemize}
\tightlist
\item
  the keeper of the physical Terminal
\item
  maintain relationships with users of the Terminal and Domain
\item
  update the Geometron server at the hosted Domain with links to the IP
  address on the local wifi network of the Terminal, as well as the wifi
  network name and password or link to where to get it(e.g.~coffee shop
  register)
\item
  post ads for money or barter on the Geometron server at the hosted
  Domain for people who ask
\item
  post on the global Geometron server when and where the Operator will
  appear, or where the terminal is set up on what network if it's
  installed permanently.
\item
  Teach anyone who wants to learn how to be an Operator, recruit new
  Operators
\item
  Tell new users about the Network, teach them to use it, how to post,
  edit, delete
\item
  Promote any kind of business or other venture or project anyone
  physically local to the wifi network area has in exchange for barter
  with that user for useful things on location(including just a place to
  operate)
\item
  Spread Network into new places by finding a location with a wifi
  network, buying a domain and setting up hosting or getting someone to
  do that and pay for it,
\item
  Domain names spread in physical space using physical media with
  depiction of domain which points back to terminal ip address, wifi
  address and password, photo of terminal and operator other physical
  media(post cards, book marks, spray paint stencils), spreading the
  physical media with the domain name
\end{itemize}

Skills of Operator

\subsection{Domains}\label{domains}

\subsection{Terminals}\label{terminals}

\subsection{Laptops}\label{laptops}

get ubuntu working under windows, install apache and php

localhost

code goes from terminal to laptop to github to
