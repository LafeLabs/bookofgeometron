\section{Trash Robot}

\subsection{What is Trash Robot?}

Everything here refers back to Ontology.  We build this thing, show how to replicate it, show how replication can benefit the replicator, which stimulates further replication.  Build things and sell them. Build things and use them. Build things and share them for mutual aid and benefit. 

Metabusiness.  SRS.  Thing.  Organic Media. Geometron vehicle.  Maker swarm. Methodology.  Trash Magic(self replicating media from trash, seed of full complete set).

The Street Network can be used to sell a huge range of goods and services.  Trash Robot is a specific collection of self-replicating technologies which can also be sold directly via the Street Network.  It includes a list of things which will be described here, which readers are encouraged to copy and sell.  But more importantly it provides a prototype for a new type of product which can be built at scale using the decentralized craft mode of production.  As soon as this book is released and we build a swarm to start living Geometron and selling the elements of Trash Robot and bartering for the services provided by the Network we will also start building totally new products using this method.  We will work to build products which have not exact equivalent on the market in the consumer economy and which would be very difficult to replicate in that space for various logistical reasons of how technology scales.   IN particular we need to explore in this chapter the economics of enclosures and how cardboard and duct tape and geometron built by crafters and immediately sold at a profit on a one-at-a-time basis differs from injection molding of enclosures at scale.  

\subsection{The Open Brand}

rainbow and googly eyes.  Soft black textiles. Felt.  Geometric constructions from Geometron.  Use of geometry in cardboard fabrication, HDPE sheets. Modularity.  Things carried in cloth bags. Blocky geometric font.  duct tape and cardboard and sticks. Things built from trash.  Images of the things: box, flag, shirt, pants, bags with robots.  The fashion brand.  

\subsection{The ArtBox}

how to build.  How to use.  The tape snake. the Trash Tie.

\subsection{Laser Cut Acrylic}

Shape set. Rulers. Protractors. Penrose Tiles. Custom prints. Spray stencils.  

\subsection{skeletron}

poles, geometry, tetrahedron, octahedron, icosahedron, tripods, flag poles, S Hooks, photos of constructions

\subsection{textiles}

the font. flags. Bags. Clothes.  Methodology. Fashion business on the street. all the layouts of the designs.  Waving the flag with the pole. Power of the flag to direct traffic to pages in the Network over the cardboard sign, legitimacy via the open brand.

\subsection{The icon token printer robot}

Build the brain. 

Build the controller. 

Build the mechanicals.  

Workflow:  image feeds, align, trace, share, print, bake, stamp, replicate, paint and sand, build into sets, share and sell, make pendants, stitch into clothing and accessories, make jewelry. Make more robots. Donate, share, and sell robots. 

Some notes on the pi version, how this can be a pi driven robot.

\subsection{Arduino Generic Shield}

This is a stub.  It expands into many technologies but we document just the board and most basic of codes.  Future technologies will reference this.

\begin{verbatim}
Trash Robot = {
    ArtBox,
    Trash Tie,
    Tape Snake,
    Token Printer,
    Terminal,
    Brand,
    Textile,
    Shop,
    skeletron,
    constructions
}
\end{verbatim}
    
\begin{verbatim}
constructions = {
    duct tape, 
    cardboard, 
    trash ties, 
    HDPE sheets
}
\end{verbatim}
\begin{verbatim}
shop = {
    ArtBox purse, 
    Trash Robot branded clothes, 
    laser cut acrylic, 
    token printer kits,
    tokens,
    pendants,
    printed bottle caps,
    terminal install
}
\end{verbatim}

\begin{verbatim}
laser cut acrylic = {
    golden triangles, 
    penrose tiles, 
    full set, 
    ruler,
    protractor,
    custom shape,
    spray stencils
}
\end{verbatim}

\begin{verbatim}
printer = {
    brain,
    controller,
    mechanicals,
    workflow
}
\end{verbatim}

\begin{verbatim}
    printer workflow = {
        build, share, sell, use printers, following instructions on scrolls on Geometron
        use Geometron server to follow the rest of this workflow: 
        image feeds,
        aligner,
        trace,
        share feed with other users, save, copy, paste, share share share!!
        load code into Arduino, print in clay tablet, bake it, sell or give away or use
        use print to create stamp, sell or give away or use,
        use stamp to create both coin-like tokens and pendants, stamping in clay, baking, using paint pen and sanding flat, sell, trade, or wear
    }
\end{verbatim}
    