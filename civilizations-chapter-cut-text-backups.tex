
When we build an economic system based on self-replicating geometric
constructions from trash and organic material, money becomes a
completely unusable system. To illustrate this, we consider a very
simple thought experiment. Suppose I take a pile of trash and transform
that into a robot which builds more robots out of trash, and which has
media encoded in it which instructs others in how to copy the whole
thing. If I and 100 people go from trash pile to trash pile and run the
replication, converting all trash piles into trash-fed robot factories,
we have created an ever expanding amount of value, with no property or
mined materials going in. A vast amount of economic activity is being
created, but without a central bank we have no way to denote all this
added value. And if we somehow created a mechanism with some innovative
new banking system to create enough currency to represent the added
value, the instant we went and replicated the system again we would find
that again, the numerical currency system breaks down, as it fails to
keep up with the replication of value.

This contradiction is precisely the situation we currently find
ourselves in. A small fraction of the population, namely the software
industry, the media industry and the finance industry, can use
replication to create value from nothing. With billions of people having
already bought smart phones or other networked devices, there is zero
marginal cost to add some useful new feature. So when an app goes from
running on just a single developer's computer to a whole software team
to a beta test group to early adopters and ultimately to a billion smart
phone users, this is replication of information, not actual production
in the sense of a factory. Meanwhile, everyone else, who are compensated
for labor or some type of physical material have to directly produce
value to get money. As long as some people create value which freely
replicates and others do not, with a finite money supply, over time a
larger and larger fraction of money will always flow to the replicators
rather than the people doing labor.

There is no way out of this trap. Not a new banking system, not a better
government policy. The trap is the numerical aspect of money itself. No
matter how fast a hypothetical banking system prints money and pushes it
out into the economy, a fully networked replication-based society can
create value faster, and continue to amplify the inequality until all
value flows to the replicators. We have seen this during the 2020-2021
COVID-19 pandemic in the United States. The government poured trillions
of dollars into ``the economy'' and at least half of it appears to have
gone directly into the personal fortunes of people in the
replication-based segments of the system. While these people also engage
in massive exploitation of labor, this is beside the point. Labor can
fight back and win some concessions from the capitalists, but as time
passes, the power will always slip away as the replication rate of pure
information-based value increases.



Some people have proposed universal basic income as the solution. This
is especially popular with the group of people who have benefited the
most from the current accelerating inequality: software engineers and
their collaborators in the professional classes. In this system, the
software people are allowed to accumulate unlimited and
ever-accelerating wealth, and simply paying out money directly to
everyone else will help the rest of humanity survive. But if a software
person is making 1 million dollars a year, do you really want a 2k/month
UBI? How about 5k/month? or 10k/month\ldots{}but now the software people
make \$1 million/month? Is that really a working economy? If this sounds
insane, look at what is happening to rents, look at who all the new
housing is built for. This is happening now, but without any UBI to
soften the blow, and it will continue to happen as long as we all cling
to this failing economic model.
