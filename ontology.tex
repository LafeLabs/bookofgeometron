\section{Ontology}\label{ontology}

\begin{enumerate}
\def\labelenumi{\arabic{enumi}.}
\tightlist
\item
  the branch of metaphysics dealing with the nature of being.
\item
  a set of concepts and categories in a subject area or domain that
  shows their properties and the relations between them.
\end{enumerate}

We consider things in the most abstract sense. From things we create
sets, which are also things. This section is a more formal mathematical
treatment of the material in the ``magic'' chapter. It is more technical
than that chapter, and does not reference magic, but does reference our
specific implementation.

This is where the Theory of Self replicating sets is described. We point
to the failures of the project of set theory in the 20th century, how it
was a dead end, and how we can navigate out of the dead end by placing
the mathematicians ourselves in the sets we describe. The whole failure
of the old models of self replicating machines(perhaps this gets moved
back to organic media chapter).

set notation, departure from ZF-C paradigm. New way of dealing with
foundations of mathematics: the mathematician is part of the sets
considered. And we continue this through the whole construction of math.

Foundational math in the paradigm of the 20th century was from ZF-C set
theory to arthmetic, to algebra, to geometry, etc\ldots{}but the whole
edifice is based on sets which follow axioms, but those sets do not
contain their makers. We create a foundation of mathematics in which the
creators of the math are always in their own calculations, starting at
the primitive level of what an ``object'' is and what a set of objects
is. We consider not only ourselves to be elements of the sets in our
universe, but every aspect of ourselves, our communities, our minds, and
our ecosystems. For example, looping back to SRS, the desire to
replicate a set is part of the set. A set might be

\{ - powerpoint slide about self replicating sets - the desire to tell
people about self replicating sets - the persuasiveness to induce others
to desire to tell people about SRS - the means to replicate the
powerpoint to the next user(e.g.~posted on slideshare) \}

From these sets we construct geometry which is used to make generalized
symbols, which are used to make all the things in our technological
complete set, which is again a part of our new approach to interacting
with abstract sets.

Random notes:

unlike normal set theory, sets can be elements of each other(we call it
``element'', not ``subset'', to erase the distintion between set and
some hypothetical non-set thing)

The structure of sets, the elements we choose to describe a set, are
completely determined by what will be the most useful for replicating
the set, for a human operator.

example Things:

The Terminal

The Geometron Terminal

The Pyramids

Audion

ArtBox

Trash Robot Icon Printer

Skeletron

laser cut shape set

laser cut ruler

custom laser cut shapes and stencils

\section{On Self-Replicating Sets}\label{on-self-replicating-sets}

\subsubsection{2020}\label{section}

\subsection{1. Computer Science and the Theory of
Self-Replication}\label{computer-science-and-the-theory-of-self-replication}

Throughout the history of modern computing machines, people have
contemplated the idea self-replicating machines. At the dawn of the
information age, John Von Neumann in particular devoted thought to the
subject, creating a blueprint that people continue to use both for
understanding hypothetical self replicating machines and for
understanding the architecture of existing computing machines. At the
same time, Alan Turing developed a similar toy model for how generalized
computing machines work, which is taught in basic computer science
classes to this day. We will not delve very deeply into these models,
but will instead present a very crude sketch of them in order to discuss
the assumptions made by computer scientists in the models they build to
understand their world.

The Turing model of a computer consists of an infinite tape of 1's and
0's along with a machine to both read and change the state of the
numbers in the tape. The string of numbers describes instructions for
the machine, which follows those instructions by moving the tape back
and forth and changing numbers from one state to another. Turing was
able to show that this toy model of an abstract computing machine could
be proven mathematically to be equivalent to any other abstract toy
model of any other kind of computer, including the complex machines
built today(hence the continued use of this model in teaching and
scholarship). This is considered to be one of the most important results
in theoretical computer science.

Before discussing this model's limitations we must say a word about the
nature of scientific models. When we investigate a thing using the
scientific method we have in principle the entirety of science knowledge
to call on, built up from a vast number of models in different fields
and sub-fields. For example, if we are presented with a rock to analyze,
a physicist might ignore everything but the crystal structure of some
prominent material in the rock and bring the modern understanding of
crystals to bear on it. The micro-biologist will only be interested in
the aspects of the rock that interact with the organisms on the surface,
while the ecologist will be interested in that but primarily how the
rock regulates the flow of water through the ecosystem of which it is a
part. As scientists we may agree that models of surface chemistry for
microbes, models of how atoms arrange in crystals, and how water flows
through rocky soil are all ``good science'', but in any given context
the model we choose to focus on depends on that context.

It is our assertion that while the model of Turing and his
contemporaries is not wrong, that it is deeply misleading because in
most cases it does not describe the most relevant aspects of the
machines we call ``computers''. Computers do, of course, compute. That
computation is described by the mathematics of computation. They also
create heat, described by thermodynamics, and are we therefore to call
them heaters? They keep time with extremely fast clocks, do we simply
call them clocks? No. But what actually are they? What is the model for
a modern computer which is most useful when trying to understand them?
In this era, in the year 2020, the most useful model for a computer is
the one that describes how they have totally changed all aspects of
global society in a relatively short time. For this we must expand the
models available, and in particular we must focus on a specific
shortcoming not just of mid-20th century computer scientists, but of
most scientists in the ``modern'' era, namely our refusal to put
ourselves and the societies of which we are a part into the systems
which we study. In some abstract sense, one can argue that we put
ourselves into the models with the role of the observer in quantum
mechanics for instance. But we do not put the role of something like
university politics into the models, even though these forces clearly
influence the science we use and hence the conclusions we draw. It is
the assertion of this paper that this blindness has become so critical
in the understanding of computers as to be wrong in a way that has real
world consequences.

So what is the problem with a Turing machine? If we look at it naively
as a physical thing, not as a mathematical toy, we see a number of
things that are not realistic, such as the infinite tape and the lack of
any meaningful human readable input or output. But these are typical of
useful scientific models: while the tape is understood to never be
really infinite, the results you can get from the machine with ``such a
large memory that it doesn't matter'' and infinite mostly don't matter,
so our toy model still works. But the critical flaw of the machine which
gives us an incorrect understanding of how computers function in society
is that it has no origin, no purpose, and no destiny. Who built this
machine? Why? How long can it run before it breaks? What happens when it
breaks? When it ceases to function, what replaces it and why?

We now live in a world where a large fraction of the computing machines
that exist live on a trajectory from a mine to a landfill of less than 2
years. During this brief journey from mine to landfill, they are mostly
used for communication, and much of that communication is marketing
information the primary purpose of which is to lead to further
consumption of similar machines. That is, \emph{in their current state}
the \emph{primary} action of these machines is rapid replication.

This is the reason we must shift the \emph{primary} scientific model of
computing machines from the Turing model to a more biological
understanding. If we seek to study a new species of life that is taking
over an ecosystem rapidly, we will always try to focus on the models
that allow us to understand that phenomenon because it is what affects
outcome. If a farmer asked us to evaluate a new crop blight and we came
back with a deep study of how carbon chemistry works because all life
contains carbon they would be very disappointed in the results even if
they are technically correct. Likewise if we are to deal with the
explosive changes to our physical world caused by computing machines we
must focus on the means of replication. We will give an analysis of
these means in the next section. However before doing that we must turn
to the history of self-replication as an idea in mainstream computer
science.

There are two main intellectual threads in this story: cellular automata
and self-replicating robots. Interestingly, it is the former of these
that actually have a vast experimental component and the latter which is
entirely theoretical. Cellular automata are in some sense a
generalization of the Turing model: they are sets of rules for
multidimensional(usually 2 dimensional) grids of numbers, generally 1's
and 0's, which follow some set of rules. These systems are simulated on
real computers, and as time advances in a program we can see fascinating
patterns of what appear to be naturally oscillating structures moving
around in a visual display of dots on a two dimensional canvas. If the
rules are created correctly, these structures can be made to replicate
themselves. The literature on cellular automata is vast and complex and
continues to be a very active field. Nonetheless, it suffers the same
flaw as the Turing model: it exists in a vacuum, with an assumed
infinite time, and no reason to exist(at least this reason is not part
of the theoretical framework used to describe them), no origin and no
ultimate destiny when it breaks. There is beautiful pure math to be
found in these systems, but no illumination of how computers function as
machines that copy themselves.

The second thread of self-replicating machines is the study of
theoretical ``robots that build robots''. This has attracted some truly
wild speculation. What is generally imagined is a totally automated
system without human intervention in which a computer controls robots
that mine minerals which are used to build both more computers and more
robots. This is then imagined to be so self-contained that it can be
used to expand out into the universe outside of Earth, eventually
consuming all things into this vast, automated, technical ecosystem
which does not need any living thing to grow. Technologists have
imagined these systems, then promised that they can be a fantastical
utopia of free things, but also warned that they can destroy the world
by consuming everything in site. For decades, theorists have written
very detailed descriptions of such systems, delving into metallurgy,
synthetic chemistry and the like to try to prove that such a thing can
be built. Most recently, the work of K. Eric Drexler and Ralph Merkle in
the 1990s pointed to a system like this built from precise positioning
of atoms in matter--essentially treating physical matter as just another
Turing machine. These theorists constructed very detailed imaginary
worlds where atomically precise machines manipulate atoms to both
compute and create, replicating themselves on a global scale. Perhaps
their day will come at some future time and such machines will be built,
but right now these models are of no help in understanding technology as
we find it today. The rest of this paper is devoted to developing both a
theory of self replication in modern technological society and in
actually \emph{using} this theory, just as the pioneers of modern
computer science used Turing's model early on, to build new things based
on the new model.

\subsection{2. Self-Replication in Human
society}\label{self-replication-in-human-society}

First, a word on self-replication in biological systems that exist
outside of human society. ``Natural'' biological organisms never
replicate themselves in a vacuum. On Earth as we find it today, all
living systems replicate as part of larger ecosystems, and the
parameters of those ecosystems are \emph{fundamental} to the overall
replication. No animal can live long enough to reproduce without a
constant flow of oxygen from plants needed for respiration. Conversely
plants need the carbon dioxide we produce in order to live long enough
to replicate. No one would dispute that a tree is a self-replicating
thing, and yet trees only replicate in collaboration with a large number
of other organisms, generally other plants, fungi, animals, all working
in concert to make the overall forest system replicate itself, of which
the trees are only a part. What we find in spite of this, however, is
that scientists outside of biology put much more restrictive rules on
self-replication, saying a thing does not ``really'' replicate itself if
it replicates as part of a larger system. Hence the flaw in computer
models: if humans replicate machines, those machines are not called
``self-replicating'' because for the overly restrictive definition of
the ``self'' of the machine they do not spontaneously replicate. But
this type of isolated spontaneous replication does not exists in nature
even for purely biological systems, so applying it to systems outside of
biology will give results that are at odds with how the biological world
works. Again, we must distinguish between models that are ``right'' and
models that describe the primary characteristic of a system under study.

In the pre-industrial societies, \emph{all} technology is
self-replicating. Historically, people make technology using the
materials found in their environment, generally from other organisms
like trees(wood) and animals(bone) or found objects of which there is a
plentiful supply. People then reproduce and teach the system of
constructing such technology to the next generation, along with enough
understanding of that technology that the young can in turn pass along
the information when they age and are passing it along to yet another
generation. This type of self-replicating technological system can exist
in stable dynamic equilibrium with existing ecosystems. Trees can grow,
be converted to boats to hunt in, which lead to cooking technology to
cook the animals killed in the hunt, and trees and game animals can then
re-grow as future generations replicate their technology.

If we take this broader view of self-replication we don't even need to
restrict ourselves to humans to see self-replicating technology. The
beaver dam self-replicates quite easily and indeed before the rise of
modern society one can imagine the dam itself as a self-replicating
entity which uses the beaver as a vector, but which transforms the
landscape vastly in excess of what one might imagine possible for a
single small animal. To make sense of a landscape transformed by beavers
it does not make sense to either study just beavers or just dams, we
much consider the self-replicating set of dam-and-beaver as a single
system. The same is true with human technology.

Moving into the very early reaches of our history as we learn it today
based on written records, the next self-replicating systems are
religions and empires. A religious text is perhaps one of the best
prototypes for self-replicating technology which can shed light on the
current state of affairs. Religious texts such as the Torah or the Koran
describe both a world view which gives people a ``why'' to what they do
which includes the replication of that text. They also include the
description of ``how'' to replicate the text, building up complex
structures of education which teach the next generation of humans what
they need to know to keep replicating the next down through the
generations. The other main replicating structure is that of the
military bureaucratic empire. An empire replicates by expanding to
incorporate more and more people into the actions of further
replication. This is generally done by force, which can keep growing by
taking more land for more mining resources and also more people to
continue to gain power to continue to expand, consuming other systems
and turning them all into that one central imperial system.

The entirety of human written history can be looked at through the lens
of these self-replicating systems, where the means of replication is the
primary descriptor of the systems. The history of what used to be called
``Christiandom'' can be seen entirely through this light. The Torah was
replicated by Jews for thousands of years, and was limited in its
replication by Jews' only replicating it to other Jews, so the growth
was limited by the biological reproduction of the people who did the
replication. Then, when Christianity appeared, the same text was
suddenly being replicated by the people of one of the vastest military
empires every built, Rome. Ultimately this replication consumed Rome and
became the Holy Roman Empire which among other things was a vast
replication machine for the scripture. Then, due to technological
advancements, it became possible to replicate that scripture
mechanically in bulk with the printing press, and we see another
explosion of change in that world where the press and how it replicated
religious scripture caused some very radical change. As Western
capitalism developed, the King James version of the Bible, printed in
bulk on mechanical presses defined the beliefs of the military empires
that then went on to conquer the globe(singling out the British Empire,
followed by the American as the most powerful of the lot). Nothing in
our world today makes any sense without this story of evolving
self-replication.

So now this brings us at last to the discussion at hand:
self-replication in regards to modern computer technology. How do
computers replicate? Just as an analysis of beaver dam replication
requires understanding the trees from which sticks are harvested and the
rivers which feed the beaver ponds, this analysis must include
externalities that are ignored by theorists, such as investors and
marketing. Modern computers exist as creations of a combination of mass
market consumerism, venture capital investment, and government research
and development mostly focused on the military. Every company that makes
computer hardware and software today is the creation of a very specific
process whereby an entrepreneur pitches a company to investors, who in
turn pitch their fund to larger institutional funds like pensions. After
they get funding, they grow using a very specific type of worker, the
modern tech worker fully indoctrinated in a certain culture. That growth
is made possible by a system of mass media that transmits the
information to consume the products to the masses outside of ``tech''.
Those masses of people both put money into the products of this creation
process and also put investment capital into the financial system that
funds the venture capital that creates all this. Nominally all this is
enabled by the ``money'' system which used to be based entirely on
mining of minerals, but is now based on some complex system of faith
more loosely based on mining. The computer systems which out-evolve
their competitors are the ones that replicate the fastest. They are the
ones that convince people to consume more and faster, and put more money
and time into the system. The venture capitalist David Horowitz has
explicitly said that in the future they are building everyone will
either be forced to obey the media on these systems or will become one
of the people building them. And indeed this is the logic of the
self-replicating machine. People like David Horowitz have to exist in
order for the machines to out-replicate other machines.

This picture presented here is of course a vast over-simplification and
the product of a fairly brief and superficial analysis. It is not the
purpose of this paper to create a full model of replication of modern
``tech'', but rather to convince the reader that \emph{such a model is
needed}, in the hopes that people will develop more accurate models that
we can use to try to gain some control over this system and ultimately
over our lives.

In summary, the simplest model for computers that I think we should
consider now is not that of the Turing machine but of the advertising
machine which exists for the sole purpose of convincing people to
consume more advertising machines. This might take the form of
presenting PowerPoint to investors, using computers to train the
workforce to build the technology, or spreading an article in the tech
press about some new technology, but in most cases it is just direct
advertising to the consumer. But in all cases the primary characteristic
which determines form is replication. This is why evolution of machines
has favored more and more of the machine being screen, with the highest
possible pixel density and color contrast, rather than maximal
computation power. Pixels are what sell pixels.

We must also distinguish between viral replication and independent
replication, although the line is blurry. Viral replication assumes a
fixed system in which the information replicates. For instance,
information can replicate itself within a commercial social media
platform like Facebook or Twitter just as influenza can replicate in a
host human, but this does not replicate the \emph{system}. The means of
replication of a system such as Facebook is in fact not replication of
content, but the whole system including venture capital, technical
labor, media to sell it, the legal framework to enforce the power of the
company, etc. It is this full system replication that we are concerned
with here, not the replication of information within such systems, known
colloquially as ``memes''.

\subsection{\texorpdfstring{3. The \emph{Potential} Power of the Open
Web for
Self-Replication}{3. The Potential Power of the Open Web for Self-Replication}}\label{the-potential-power-of-the-open-web-for-self-replication}

What is the Web? The Web is not the Internet. The Internet is a network
that connects almost all computers in the world, both physically and
with some software protocols. This network traces its origins to the
network of military and academic(but military-funded) computers that
emerged from ARPA back before the modern commercial Internet, going back
to the end of the 1960s. The Internet can in principle be used to
exchange any information and treats information in the same abstract
sense as in the models of theoretical computer science discussed above.

The World Wide Web was initially the creation of one person: Tim Berners
Lee. It was initially created as a directory for the large science
institution Lee worked for(CERN). The Web works beautifully on the
Internet and the Internet is what made it huge, but it is not the same
thing. The Web is a system for encoding information for and by humans to
communicate with other humans. It includes human readable code designed
to create universal documents which link to each other and can contain
images of all kinds and text in all languages, creating a sort of
universal document in a universal language in which all of humanity can
communicate. While the modern Web is commercialized and increasingly not
open to all users by default, this is a choice we have made that can be
un-made. In principle any computer \emph{can} be both a web server and a
web browser. If we call the ``open web'' the collection of all web files
which are openly viewable to anyone connected to the Network, the open
web can in theory physically grow to include every computer in existence
using the existing physical telecommunications network.

Let us now estimate the size of this potential Open Web network. We
suppose that given the multiple billions of smart phones, laptops,
servers in server farms, embedded systems, supercomputers, etc., that
the total number of potential web servers is of order 100 per person. We
then round human world population up to 10 billion, and estimate that
there are 1 trillion total potential servers on the Open Web. Given that
even a modest cell phone has a few gigabytes, but many servers and deep
storage computers have terabytes, we can round up a little and say the
average server can host 10 gigabytes of data. If a file like this
one(this paper, the one you are reading rightnow) is 100 kilobytes to 1
megabyte, let's round to 1 meg and say there are 10,000 documents like
this one(they could in theory all be math papers) for each server. So
the total number of documents per person is 1,000,000. But we as a
society \emph{share} these documents, so in some sense what we all have
is not 1 million but 1 million times 10 billion or 10 quadrillion
documents(10,000,000,000,000,000). The informational universe in which
we construct this new mathematics consists of this network of 10
quadrillion linked documents.

This universe of information exists on web servers which can in general
be made to run code that edits and replicates the documents. Thus
\emph{every} document in this universe of information can self-replicate
and be edited in situ. If this is all on the Open Web with code that can
be edited by anyone on the Web, all users can constantly edit all
documents, so potentially we have 10 billion people all simultaneously
editing 10 quadrillion files all of which are able to instantly
self-replicate from any node on the Network to any other Node. This vast
network effect can create power in the same what that billions of brain
cells with massive interconnection, creating a document of greater power
than any that has ever been possible before. The power of such an open
system will be so vast that it will make no sense to have any private
data. Without any property on the Open Web, things can replicate freely,
and the increased value will be so great that it will consume private
property online. This evolution will be physical as the value to the
physical caretaker of a physical web server becomes greater to
participate in the Open Web than to keep it in the commercial web. Note
that if we try to simply write down the number of ways that these
documents can point to each other to self-replicate, since each document
can replicate from \emph{any} other document in the collection of
documents(and any number can replicate from any one other document), the
number of ways they can be structured is the number of documents to the
power of itself. This is 10 to the 16 to the power of 10 to the 16. 10
to the power of a few hundred is already exceeding the number of
estimated protons in the known universe. So one can make similar
arguments about this system as people make in regards to quantum
computation systems: we can even for a very small network build
something that is totally impossible to simulate on a classical
computer.

The power of a fully self-replicating and evolving Open Web on this
scale is that documents can describe the replication of \emph{physical}
things, and the replication of the documents can include replication of
the things. If things we use in our lives are replicated rather than
purchased or mined, it changes the basic assumptions about what value
is. Note that like ``set'', ``thing'' is used in a maximally general
sense to include things like ``a feeling of awe at the largeness of a
tree'' or ``the tendency of cats with white fur to cause a change in the
appearance of black clothing''. The word ``thing'' is used here as a
placeholder for \emph{anything} which human language can be made to
describe or point to.

In order to build these documents we must first define the idea of what
exactly a self-replicating document is, and how it fits into more
general concepts of self-replication. To that end we will take an
excursion into the math known as set theory, which is the next section
of this paper.

\subsection{4. Self Replicating
Sets/Documents}\label{self-replicating-setsdocuments}

\subsubsection{Motivation and
definitions}\label{motivation-and-definitions}

Set theory, is the mathematical study of sets. Sets are simply
``collections of objects''. The idea of a set as a collection of
``objects'' considers the idea of the ``object'' at a level of
generality perhaps shocking to non-mathematicians. ``Object'' here can
mean \emph{anything}. Mathematicians generally mean by ``any object''
any object which a mathematician might talk about. However in principle
it can be anything that anyone might talk about(as we seek to generalize
these ideas beyond mathematics). For the purpose of this work we well
define a generalized object to be anything which human language can
possibly describe. Any word, symbol, or collection of words or symbols
which point to something--that something is an ``object''. And a ``set''
is just a collection of such general objects.

The notion of a generalized object is familiar to modern computer
programmers who use the idea of ``object oriented programming'' to
create generalized objects which are used to build linguistic handles in
human language on computer programs. Thus a modern programmer might
define something abstract like ``shopping-cart'' for an e-commerce
website, and then that object will have properties like ``list of
objects'' and ``total price''. We choose to take the path taken by
foundational mathematics and have our basic concept from which all other
concepts will be built be the collection of objects(which are themselves
objects) be the fundamental idea.

In order to understand the motivation of this work it is necessary to
trace very briefly the history of set theory. Through the end of the
19th and beginning of the 20th century there was a vast effort by some
of the most brilliant mathematicians in the world to construct a
universal mathematical theory base on the theory of sets and symbolic
logic. Axioms were proposed, used to prove things, argued about, and
re-written. The goal was to base \emph{all of mathematics} on the axioms
of set theory, and to go from there to a universal system of truth in
which statements may all be proven to be true or false.

People like Bertrand Russel pointed out that such systems can create
paradoxes that make it impossible to create a self-consistent system of
logic/truth/math. This paradox can be summarized by considering the
``list of lists that do not list themselves''. The list defined here is
a list of lists. Is this list on itself? If it is, it cannot be by the
definition of itself. If it is not, it must be by the same reason. In
spite of having publicly stated this paradox, Russel and his co-author
Alfred North Whitehead wrote what is now considered a seminal work in
mathematics \emph{Principia Mathematica}(not to be confused with a book
by Isaac Newton of almost the same name), which attempted to create such
a universal basis of mathematics. While the achievements of 20th century
set theory, logic and analysis are fantastic and useful, they ultimately
failed in their goals, and this was proven mathematically by Kurt Goedel
in 1931.

In the post-Goedel world we should take for granted that no universal
logical construction can be built which defines truth and falsehood
without contradiction. Goedel's proof presented a fork in the road
intellectually. We could have used this as the sign to go back through
mathematics, accept contradiction as part of life, and build a math
based on desired outcomes. In some sense this is what society did by
mostly ignoring the work of most post-war mathematics(with some very
narrow exceptions like number theory for cryptography). While
professional mathematicians took the opposite fork, building
increasingly complex systems based on each other, where a vast tower of
ideas linked by formal logic built up from the axiomatic set theory of
the early 1900s to create a bridge to nowhere. The sole purpose of most
mathematical concepts and theory today are to advance the career of
working mathematicians. We forget, both outside and inside mathematics,
that people used to believe ideas in math actually \emph{mattered}. We
also forget that mathematics has for thousands of years been one of the
most powerful tools the human mind has for understanding and interacting
with our world, and indeed mathematicians have traditionally played a
central role in the largest power structures throughout history.

Having proved that a universal truth machine cannot exist, we may now
abandon the project of early 20th century mathematicians such as Russel
and Whitehead and proceed to reconstruct axioms of set theory based on a
\emph{desired outcome}. This system will not be judged on its ability to
prove theorems, eliminate logical contradictions, or get tenure for math
professors. It will be judged \emph{only} on its ability to improve the
human condition. It is time, finally, after almost a full century, to
inherit the true legacy of Goedel.

Right now all of humanity is locked into one giant self-replicating set
which has as elements all of industrial society. The purpose of this
work is to create a set theory which enables people to construct sets
which create the maximum amount of human freedom. To that end, we seek
to make sets have elements that are defined as generally as possible and
also which always have the \emph{desires} of the creator of the set as
an element. When we move forward replicating these sets in the new
society we build, each act of replication involves also replicating this
desire. We therefore only replicate that which transmits a desire that
we consent to and agree with for some reason.

Let us make some statements here about the sets we want to define. We
define self-replicating sets as sets which contain as elements the means
to replicate themselves. We also state that since our goal is to create
the objects of our desire with our mathematics that whatever that desire
or intent will always also be an element of the set. We maintain the
tradition of both formal set theory and object notation on computer
science and define sets in writing by listing elements separated by
commas and contained in ``curly braces'' ``\{'' and ``\}''.

We also state that in general the sets we will construct will have a
primary element, the replication of which is the purpose of the set.
There might be many other elements and subsets which are needed for
replication, but the \emph{primary element} is defined as the element
the replication of which represents the \emph{primary intent of the
creator of the set}. We thus make the human will, desire, or intent a
fundamental element of our entire set theory. To distinguish our set
theory from that of mathematics as it exists today we coin the term
``set magic'' to be the theory of sets which contain both the desires of
the creator of the set and the means to replicate the entire set. This
is loosely based off of the quasi-secular use of the word ``magic'' from
both chaos magic and Thelema magick to indicate the attempt to impose
the human will on the world we find around us.

Thus a way to express the most general possible self-replicating set
from our newly defined set magic is:

\begin{verbatim}
set = {
    desire,
    object of desire,
    means to replicate this entire set
 }
\end{verbatim}

Note also that in order for the whole set to replicate, the desire of
the creator must also replicate. This is what in our existing system is
called ``marketing'' and ``sales''. Without first convincing another
mind to share our desire for replication, it will not happen. The power
of the open web is thus not just about replicating documents but
replicating the desire to replicate documents--in modern parlance, just
marketing. The commercial web has proven excellent at this, and the open
web has the potential to vastly improve on that.

We further state from the outset that all self-replicating sets have
externalities. These will be elements or subsets of elements which draw
on the resources outside any given ``closed'' system in order to
replicate. Thus rather than attempting to build a conceptually closed
system and then finding that it is not closed as did earlier set
theorists, we accept that we are building a less formal construction
that will always have an externality and that we must describe what this
is in order to properly define a set. The number of degrees of freedom
of this externality is what limits the overall degrees of freedom of a
system. For example, if an element of a set is ``text reading system''
that can be on any of many different technologies. Whereas if something
has an externality ``lithium ion polymer batteries'', the entire system
is reliant on that one technology and any threats to large scale
extraction of lithium from the Earth are threats to the whole system. We
thus seek to constantly struggle to improve on the externalities, with
the ultimate goal for them to be in equilibrium with the living Earth.

A self-replicating document is an example of a self-replicating set(a
``document'' is just a collection of symbols, hence a set whose elements
are symbols). This document is created as itself a self-replicating
document according to the prototype we propose for the whole system. We
now set forth to define the set which is this paper in theoretical
terms, then to describe all the subsets which together make for a
self-replicating set which can be evolved into other self-replicating
sets. We now proceed to formally define the set which is this document.

\subsubsection{This Set}\label{this-set}

The prototype self-replicating set we define in this paper is itself
this paper, and is formally defined as follows:

\begin{verbatim}
On Self Replicating Sets[This Paper] = {
    The desire of the author to describe self-replicating sets,
    A description of self-replicating sets,
    The means to replicate this set
}
\end{verbatim}

The first of these will always be a part of the system of sets we are
constructing here: desire. All sets are defined with an element which
maintains the desire of the author/creator/artist, and this is
maintained as sets replicate to ensure that sets only replicate with the
intent/consent of someone. Furthermore, we separate the thing being
replicated from the means to replicate it. Elements are generally
themselves sets which are broken down into more finely defined elements.
This paper is part of the ``description'', but what formally is ``this
paper''? We seek to define it as a set, or a collection of elements
contained in ``description of self-replicating sets''. This includes the
following elements:

\begin{verbatim}
Description = {
    narrative structure,
    definitions,
    digital text document,
    pdf document
}
\end{verbatim}

Now again we break off the replication methods from the thing being
replicated. Replication is to happen on the Open Web. That is, it must
have self-contained and self-replicating code that can copy itself from
any web server to any other, and be edited after being copied, then
copied again from the new instance so that the information is totally
decentralized and evolves naturally as it's copied and edited. We also
need this document to include instructions in human language(English in
this particular instance) on how to replicate the whole system, by
either buying a domain and setting up paid web hosting or building a
physically local web server to server the files. This document will
contain that information. Furthermore, the replicator set must include
the other media that are used to replicate the whole set, including
content on commercial social media.

\begin{verbatim}
Replication = {
    code replication,
    server replication,
    pdf replication,
    in-person pitches and classes and discussions,
    media: email, post cards, posters, signs, commercial social media
}
\end{verbatim}

\subsubsection{Code and replication of
code}\label{code-and-replication-of-code}

The elements of the self-replicating code that can propagate this
document across the Open web are as follows:

\begin{itemize}
\tightlist
\item
  \href{php/replicator.txt}{replicator.php}
\item
  \href{data/dna.txt}{dna.txt}
\item
  \href{php/filesaver.txt}{filesaver.php}
\item
  \href{php/fileloader.txt}{fileloader.php}
\item
  \url{README.md}
\item
  \href{php/editor.txt}{editor.php}
\item
  \url{index.html}
\item
  \url{pageeditor.html}
\item
  \href{php/dnagenerator.txt}{dnagenerator.php}
\item
  \href{php/text2php.txt}{text2php.php}
\end{itemize}

All of this code can be edited using the program \url{editor.php} which
runs on any server that has \href{https://www.php.net/}{php}
installed(most web servers). The file dna.txt represents this list of
files, which replicator.php uses to fetch them and copy them to a new
server(see below). README is the text of this document itself, and the
name of that file is set by the standards used on
\href{https://github.com/}{Github} so that by default any
self-replicating document that's put up on Github has its content
readable immediately. The format of the README file is by default
\href{https://daringfireball.net/projects/markdown/syntax}{Markdown}.
The save and load scripts are required to edit files on the server, and
pageeditor is the page that uses these files to edit the main manuscript
README.md. All php files are stored as .txt files so that they can be
readable and easily accessed from the open web. The program text2php.php
copies all the files in the php directory to the main web directory and
changes the file extension from .txt to .php so that they can run. The
file editor.php edits the copies in the /php directory, and then running
text2php makes those programs executable.

\subsubsection{Server replication}\label{server-replication}

In order for replication to take place from server to server we need the
ability to ``colonize'' new web servers with this code. This is done in
any of several ways. Right now the main way is to buy a domain(usually
about 10 dollars for the first year), pay for web hosting(5-20 dollars
per month), then put the replicator program on the new server and run
it. The second method used is to put a web server on a Raspberry Pi, a
computer that can be bought for as little as 35 dollars and fit in a
pocket which is excellent for serving web files over a local network.
This allows for a grey area open web that is open to anyone on a local
wifi network, and can see the rest of the Open Web but cannot be
accessed by users outside that wifi network. In either case, the
replication of the server consists of placing the file
``replicator.php'' in the main web directory of the new server, then
pointing a web browser to {[}your new servers web
address{]}/replicator.php. This will then cause the program to run,
copying the rest of the system and linking to the main page which will
display the newly replicated document.

The technical details of this process must be described briefly here in
order for this document to truly self-replicate. We recommend buying a
domain and getting shared hosting on
\href{https://www.dreamhost.com/}{dreamhost} because they have proven to
work with this code and are affordable and not a scam. One can also use
any of several free web hosting options, including
\href{https://www.000webhost.com/}{000webhost}. In both cases you will
find a file editor screen(this can be a little frustrating to find but
always exists) which will allow you to create new files and edit them.
Use this to create the file replicator.php and copy the code from
\href{php/replicator.txt}{here} into it, save it and close it. On the
Raspberry Pi, replication starts by making a new flash card with the
operating system on it. One must then
\href{https://www.raspberrypi.org/documentation/remote-access/web-server/apache.md}{install
the web server and php language using this set of instructions}. After
that one can move to the main web directory, copy all the files, change
permissions, delete the existing page, and run the replicator:

\begin{verbatim}
cd /var/www/html
sudo rm index.html
sudo curl -o replicator.php https://raw.githubusercontent.com/LafeLabs/thing/master/php/replicator.txt
php replicator.php
sudo chmod -R 0777 *
hostname -I
\end{verbatim}

Once the whole thing has been copied, to have the \emph{next} copy be a
copy of this copy and not its predecessor, use the
\href{editor.php}{code editor} to edit the file replicator.php so that
it points to the global url for the dna on your new page, not the
previous one. This is done manually for now. So if you do not do it, the
next copy will be not of the new document but of its predecessor. Note
also that once this system is installed anyone can run any code on your
server, so no private or personal data of any kind should ever share a
server with this system. This system assumes that no private data exists
unless it's on a physically isolated wifi network, and it must remain
separate from the private Internet(especially anything with e-commerce,
as stealing financial data would be trivial if that is ever done).

Another good practice as one works with files on the open web is that
since we must assume all files might be deleted at any time, but we want
the current copy to remain readable not just by us but by all users on
the Network, we constantly back up text to free and open but
non-editable paste sites like
\href{https://pastebin.com/}{pastebin.com}. To fork the whole code, one
can edit on a live Open Web server, back up to a pastebin, and copy. But
one can also create a github repository, copy the code locally to your
hard drive, edit it on there, run dnagenerator.php to make a new dna
file, then point your replicator to your github repository so that many
copies can be made without the original being corrupted(using a hybrid
between the password-protected space of Github and the Open Web which
copies files from there). Note that while one can use any code editor
for editing the local copy, we can keep a consistent system with the
Open Web servers by running php -S localhost:8000 at the command line(I
assume people who fork the code in this way know what this means) and
connecting a browser to http://localhost:8000 to edit \emph{in situ}.

\subsubsection{Pdf document}\label{pdf-document}

As useful as the Open Web is, it is also useful to have documents in a
traditional format which is compatible with physical paper printers to
make copies we can carry outside of digital readers. To do this we have
several options. We can print from the browser(which will look bad), we
can convert to Microsoft Word which will corrupt the file and also look
bad, or we can use the LaTeX system which will look great but take more
work. For this initial instance we focus on the LaTeX version. As this
document replicates and evolves hopefully more skilled users than the
initial author will make smoother systems for conversion but right now a
combination of the Haskell library ``pandoc'' and manual editing of the
output file are the easiest way to convert from markdown to LaTeX. A
document produced in this way is included in the replicator of this set.

\begin{verbatim}
pandoc -o paper.tex README.md
pdflatex paper.tex
\end{verbatim}

\subsection{5. Generalization and Social
Implications}\label{generalization-and-social-implications}

All ideas which we desire to propagate on the Open Web need all the
typical means of use of media to try to convince others to replicate the
document. In some cases we seek to explain the whole thing in a tiny
capsule of information, as with the ``elevator pitch''. In other cases
we seek to show by example the power of these ideas. The media we expect
will be used in the spread of this system include \emph{all} media in
the most general sense, including physical things like machines which
carry various writing on them. Any physical thing can have a domain name
on them which points to a document which describes how to replicate the
physical thing. This extremely generalized definition of
self-replicating document(as a type of self-replicating set) means any
physical object can be thought of as a self-replicating document.
Combined with the ``Open Web'' defined above, this can create an entire
universe of useful things we can use to make up the fabric of our lives,
building sets to live in which are independent of the existing
industrial order.

Ultimately if we can build sets that we have the capability to evolve
based on our desires, we can push that evolution toward what we call a
``technological complete set''. That is a set which describes a full
self-replicating system that we can live in, which can exist in
equilibrium with its environment just as pre-industrial societies did
before the whole world was consumed by one very destructive set. In a
complete set, the people who live in that set(we place ourselves
conceptually into the set) have everything they need for a good life
such as medicine, abundant food, clean water, the ability to live in a
comfortable temperature etc. In addition, a set is not complete if it is
out of equilibrium: if a set requires constantly destroying things and
not replacing them to exist it is not complete. As we look to the future
this is possible in a way that is totally different from what was
possible before the rise of industrial society. A future complete set
based society has no reason to mine, since the quantity of material that
has been extracted from the Earth and processed into very ordered
structures is more than enough for a large human population to live on
indefinitely.

This Technological Complete Set does not need to be designed and built
by any one person or group. All that is needed to achieve such a set is
to build sets which people have the capacity to evolve based on desire,
and to impart into a group of people the desire to achieve this
set(given the assumption that such a set is physically attainable). This
document is meant to describe such a set, but also to serve as a seed
which the author hopes will evolve in such a direction. By itself it is
probably insufficient to build a large complex system, but it provides a
prototype for a number of other self-replicating sets which we will
construct and replicate over the Open Web. It is the nature of such sets
that as they are created and released into the wild, they will all build
on each other with network effects and that a small amount of
exponential growth early on can create very large effects as the system
evolves.

What is next in this program? The self-replicating sets which are
currently being created and released are largely about replicating
symbols. Symbols and logos play a very powerful role in how our minds
process the world around us. The ability to create a self-replicating
symbol which has some intention imposed on it is one of the most
powerful forces in our world today. This is the power corporations wield
with their brands, logos, and marketing messages. By building systems
for very rapidly creating symbols, giving them meaning, and replicating
them, we empower the masses with this same power. The media for the
symbols includes artistic tools for physical media creation(stationary,
wall art, postcards, signs) and digital media creation(vector graphics
of simple geometric logos). From this power, we hope to build a fire
that consumes the media landscape and transforms the nature of human
existence on this planet.
