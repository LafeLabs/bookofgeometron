
In a consumption-based civilization the purpose of all media is to stimulate consumption.  This can come in the form of advertising, corporate propaganda, or the legitimization of the imperial power required to keep mines and long distance supply chains operating.  In the age of digital media we find that the hardware itself is also a large component of how it facilitates consumption, with planned obsolescence creating a stream from mine to landfill unprecedented in human history.  A constant race to build ever more exotic materials and technologies into physical media devices creates a vast suction across the planet, forcing every corner of the globe to exploit anything that can be mined for digital media hardware from cobalt to lithium to be exploited as fast and widely as possible.  

This state of affairs creates a powerful opportunity for a new type of media.  The fact that the existing system pumps out this constant stream of new objects with all the stuff needed for advanced information technology creates a resource which can be used to build new hybrid technologies designed to incorporate scavenged parts from the discarded tech.  The tech industry is now working on turning out \emph{trillions} of ``internet of things'' devices--objects which have built-in networking capability and could in theory serve as web servers.  Based on how these industries are structured we know that these will all be designed to fail on a very short time scale, and what they are selling today will be in a landfill in less than 5 years.  

It is worth marveling at the scale of consumption built into the current digital media system before discussing the alternative.  People walk around with screens in their pockets, the sole purpose of which is to manipulate them into consuming more.  Those screens are built on a technology which uses the most exotic materials known to humanity, extracted at great human cost from every corner of the globe.  And then they are forced into obsolescence within months in some cases by a software industry which is based around the idea of planned obsolescence.   Having pushed their way into the pockets, homes and workplaces of something like half the humans on Earth, the industry is now pushing to put their devices in things which have no reason to be part of this network, like toasters and juice makers.  The sheer insanity of this is hard to wrap ones head around, but because the entire media is controlled by this industry it is hard to even articulate in public how insane this is.  And it is getting worse very quickly.  The need to replace this parasitic monster with media which serves the needs of humanity could not be more urgent!

What we want from a media technology to build  our new trash-based civilization on is to replace consumption with replication.  We are now constantly buying new machines built from mined materials which constantly tell us to consume more things.  Consumption-based media forms a consumption information loop.  We want to form a replication-based loop, where the media is built from trash and contains the information required to replicate itself.  We call this ``organic media'', because it behaves like a living thing.  In fact it effectively \emph{is} a living thing.  If we built closed loop systems in which humanity is re-using material again and again forever in order to live in harmony with the world around us, it makes sense to think of us in combination with our media and the ecosystems we live in as a living system.  

This idea of ``organic'' media is in direct contrast to ``viral'' media which dominates in social media today.  In viral media, information replicates, just as viruses replicate themselves inside a living organism, but the this is always happening in a space defined by a fixed media entity.  Social media platforms encourage information to replicate as fast as possible within their systems, as that costs them nothing and induces more people to keep coming back to their platform to be manipulated by their advertisers.  But if someone tries to replicate the media platform itself by for instance trying to start a new platform, they will do anything they can to stop them.  Media today can be viewed in biological terms as an apex predator which kills everything in the rest of the ecosystem and is full of viruses.  

When we say we want media to be organic what we mean is that we want the media platform itself to replicate.  Just as each new tree or squirrel in a forest really is a whole new instance of ``tree'' or ``squirrel'', with no central entity controlling them, we want each new instance of our system to be self-contained.  We want it to be able to replicate itself right where it stands, with no outside input from some central system of any kind.  This is only possible because of the waste of the existing industrial system: all the materials needed for advanced information technology are sitting in trash bins, dumpsters, closets, and landfills within walking distance of wherever you are reading this right now.  All you need to build a whole new media ecosystem from scratch is information: the information required to gather the people and materials required to build it.  If the system you build tells people how to do this, it can freely replicate across the whole world without any central infrastructure.  

It is worth noting that building this is hard. Modern digital media technology is designed by hostile engineers to be as hard as possible to fix, modify, or use for anything other than consuming advertising for a few months before it goes to the landfill.  Those machines are built by vast teams of well funded groups with extremely specialized technical skills.  It will take a concerted research effort to fully replace the existing information technology system with a free one built from trash.  In a later chapter, I will discuss how we can do this by using a different system architecture in which the purpose of the whole system is displaying of a specific class of documents based on the software presented here.   But for the time being, in order to launch our new media system, we will rely on existing off the shelf hardware which is still part of the consumer system but not the main commercial advertising-driven part.  

This book is therefore doing two things in regards to launching this system.  It is launching a new social media platform based on using the Raspberry Pi as a local web server used over local wifi networks, and it is laying the conceptual framework for building a whole new information technology system from the ground up on new principles.  Just as technology people in the existing system refer to a ``technology stack'' we are building a whole new ``stack'' in the sense of a collection of technologies which are all related by a chain of increasing or decreasing abstraction or proximity to the user which work together to make our system work.  In this work, we describe the whole stack.  We launch fully functioning software and hardware for parts of it, and describe how we will build the other parts that require more work.  

The most important part of the current project presented here is that it work for its purpose which will support the rest of the development.   This means that the technology has to work to distribute new technology of all kinds built from trash which people need or want, which can be freely replicated.  We use this partially consumption driven media system to launch self-replicating media systems which really are built from trash as a demonstration.  Our metric of success will be how this self-replicating media technology replicates and evolves.  If we can make it replicate by making things people want, and make it evolve by creating a strong incentive for people to improve it, the system will naturally evolve into the one we need, which no longer requires any input from the mine-based system to function anymore.  

Just as a relatively small number of people a few hundred years ago sending each other letters built the basis of the current explosion of technology which led to the existing world order, we believe that a small number of people with new ideas today can build a much faster explosion of information which consumes the existing world of consumption and replaces it with locally closed loops of material in a single generation.  We also believe that this is of the utmost importance to do as quickly as possible.  The existing system is killing us.  It is destroying the natural world, needs constant warfare to function, and is increasingly driving anyone outside the technocratic elite into extreme poverty.  

So how do we actually build this ``organic media''?  We start by looking at living systems, both individual organisms and larger systems like forests.  The most fundamental thing life does is replicate.   This will probably get tedious for the reader, but replication is the thing this work will come back to with relentless repetition but that relentless repetition of replication is precisely what makes life work.  A living system is a system of thing which all also replicate.  Living systems replicate over all scales: forests replicate, but so does the RNA and DNA in each cell of each organism in the forest!  We will also build our systems this way: many components make up systems from tiny scraps of code or single cutouts of cardboard up through whole vast industrial fabrication systems, and we want \emph{all} of them replicating.  Again this is in direct contrast to the existing system in which small parts like shared memes on a platform are supposed to replicate but the company itself is designed around non-replication.  

Another property of life is that it is an independently evolving thing.  Because organisms have an independent life, they can change in much more unpredictable ways than centrally controlled systems like a large corporation, government or non profit(this includes open source software projects with a central code base that all instances are copied from).  

Finally, all life dies.  In order for life to work we need the cycle of death to be natural.  Just as fungi in a forest turn logs into soil we need the destruction of all things in our system to be natural.  This is again in direct contrast to the existing system in which all media is built out of a company which is designed to grow forever and never die. 

In order to build our platform then we want to write down a set of rules which will guide all of our work.  These are nine rules:

\textbf{Everything replicates.} This is the most fundamental law.  It is what makes life alive.  And it is what makes media organic rather than viral or parasitic.  This means that all our software contains code to replicate itself without any reference to a central code repository.  The code on a server in a coffee shop can directly replicate to the laptop or phone of every person in that coffee shop with no connection to the rest of the Internet at all.  All our hardware is built into some kind of media which describes its replication.  

\textbf{Everything evolves.}  All things can be edited by all users.  To be in contact with a thing, be it a file or a physical machine is to have the power to alter that thing totally.  There are no ``users'' or ``engineers'' in the sense used today.  Some people will choose to edit things more than others but everyone \emph{can} edit all things.

\textbf{Everything dies.}  All things can be deleted or destroyed by all people.  This is particularly important for files.  Much of the power structure we are trying to destroy rests on information we are not allowed to destroy, from the intrusive and parasitic industry of buying and selling personal data through the constant advertising we are not allowed to turn off.  Also, in order to be able to stop harmful information, we empower every single user with no exception to be able to delete every single piece of information they come into contact with, with no exceptions. This is less destructive than it sounds. Because our network is all physically local, no central bad actor can wipe out the whole network.  If all networking is at the level of a wifi network, and they are all constantly being destroyed and rebuilt anyway, the cost of universal destructive power is outweighed by the benefit of people being able to stop bad information without any appeal to authority. 

\textbf{No property.}  As discussed in the previous chapter, the idea of property is not compatible with a civilization based on self-replication.  Since the idea of property fails at scale in a replication society, and since our goal is to scale, we dispense with it immediately and build infrastructure which not only has no intellectual property but where the physical web servers are not owned by anyone.  Initially we will have to spend money to buy parts to build them, but as soon as they are built we will release them to whoever we think will get the most use from them, along with instructions for them to do the same, passing all infrastructure along to wherever it gets the most use.  Building network infrastructure without property for the benefit of our communities means that our incentives are now to find whoever has the most need, identifying their needs, and using our technology to serve those needs and fast and directly as possible.  If people benefit from the systems, they will naturally be able to replicate, which will further replicate the non-property technology.  Initially this means building Raspberry Pi based web servers and giving them away to the people with the greatest need.

\textbf{No money.}  This is connected to the rule against property.  Initially of course we will need to spend money to buy parts, and will need to have users make money to support their near-term survival.  But as we scale up and get more and more basic needs satisfied by technology built from trash, we want to have the elimination of money be the direction we are headed from the start.  This is not as far fetched as it sounds.  As will be discussed in the next section, barter will be an incredibly powerful tool for scaling.  Our network can provide huge benefit to very powerful and wealthy people, and if our people with the most need are dispensing this benefit, we will be able to barter the things we need to scale directly.  When our network helps a business with a lot of unused space to make money, they can let us use their space.  When a business person makes connections using our network which make them money, letting us scale our software up on some of their servers will make economic sense to them.  We will be able to barter our value as network builders into the things we need for personal survival like places to sleep and food but also the things we need to scale our technology like access to labs and machines.

\textbf{No mining.}  Our long term goal is the global elimination of all mining.  This includes the whole natural resource extraction industry such as oil and gas as well.  This cannot happen overnight, but we don't need it to.  Every single mined component we replace with one from a dumpster or landfill takes a little bit of energy and power out of the mining system. if we can remove power from them in a way which self-replicates, our system will simply consume theirs, and mining will be eliminated in a generation.

\textbf{Everything is physical.}  This is almost a circular statement.  What does it mean for a thing to be ``not physical''?  This is a statement of belief.  We emph{believe} that the idea of information which is not physical is meaningless.  All information has a physical manifestation, be it charge on a transistor or bumps on a CD.  This law is important as a vocal rejection of any theory of how machines works which states that information or data can exist independently of its physical existence.

\textbf{Everything is recursive.}  One of the most notable properties of life is its constant self-referencing.  Billions of DNA strands in each individual body of a large organism all contain a whole copy of the information required to replicate the organism.  We see information which points to information which points to information.  Life is very self-referential and involves in an abstract sense functions which call themselves constantly.  RNA stores instructions to make molecules which replicate RNA, and so on.   This law is to remind us as creators of technology to be \emph{constantly} thinking of ways to make things point back to themselves to replicate. 

\textbf{Everything is fractal.}  This is another property of living systems that we take for granted but which we either ignore or make very crude imitations of presently.  Centralized systems of control create technologies which are flat in scale: we build microchips with nanometer precision across millions of nanometers(mm) of scale and so on.  In contrast, living systems are fractal in scale such the scale of ``error'' required to cause catastrophic failure scales with the size of the system.  We assume that patterns will repeat again and again at different scales, and expect that our technologies only need to be precise at the correct scale for any given sub-system.  This has very specific implications for fabrication which will be explored elsewhere, but as a law we mean that we must simply always have ideas of the fractal nature of living systems in our minds as we create new things in our new trash-based civilization.


With our goals and laws of operation stated, we are now finally ready to delve into more detail into what we are actually constructing with this work.  This is a local network based on a web server loaded on a Raspberry Pi.  The Raspberry Pi is a computer on a circuit board about the size of a deck of cards which typically costs about \$50.  It needs some peripherals including a screen, keyboard, mouse, battery, and memory card, all of which makes it about \$200 for a nice, easy-to-use, portable and self-contained system.  It only requires a few commands which are easily copy/pasted to install the whole functioning Geometron system on a new Raspberry Pi.  It is also modular, and if portability is not needed it can plug into the wall, borrow a keyboard from another system temporarily, and display on a TV, making a non-portable system cost just the value of the board(\$50).  

It is also important to note that this is all modular, easy to buy from many sources, and involves parts many people already have lying around.  The Raspberry Pi is widely marketed as a hobby tool and a STEM education tool, but its use case is not always clear.  Therefore a large number of people own them but do not use them.  They are often sitting in drawers in someones home office or a under-used maker space gathering dust.  If we have a use for them and can provide useful services to people it should be possible to directly barter with people who want to support our network who will be willing to donate the Pi boards to our network, where they will become non-property and be distributed to those in need.

The actual software we run on the Pi is what is described in the bulk of this book.  It is all designed to run in a web browser. Any web browser.  So if a Raspberry Pi running the Geometron server software is on a wifi network, all the programs and documents on it can be read, used, edited, replicated and deleted by anyone connecting to that wifi network on any device be it a phone, laptop, tablet, or another Raspberry Pi.  Also, the pi itself can serve be used in the same way as any other device on the network.  

We must also note an important condition for the pi to be a non-property computer.  In order for the pi to be able to freely be shared among the people, it cannot have any personal data on it which causes someone loss if it is read by another person.  That means we must never log into private networks like gmail, facebook, or more sensitive things like bank accounts ever on the system.  In order for a free and open system without property to function, it must be kept separate from the property based networks.  This media platform exists for the sole purpose of free sharing of documents.  Any document we do not wish to share we do not put on it. 

Also, there are no ``users'' on this network.  This is a network of documents, not users.  There are no logins, no passwords, and no databases.  User data is not harvested for profit because we simply do not generate the type of information which is considered ``user data'' in the existing systems.  

We take as an axiom in the development of this system that documents intended for free sharing are of greater value than private documents, and that the network effect as the universe of non-property documents grows will exponentially increase their value to people until our network out-replicates the existing ones.

Our network is based on sharing several specific types of document which are encoded into the software.  We share ``scrolls'' which are text documents, ``maps'' which are like presentation slides or memes, ``feeds'' which are essentially lists of information, self-replicating applications of all kinds(using web based code that runs in a browser, never native code), and generalized Symbols using the Geometron geometric programming language which takes up much of this book.  Taken together, these systems of document creation and replication will allow us to describe and replicate any technology of any kind.  

This network will be distributed physically, over the Street Network described in the
 next section.  We will initially try to get servers to people who are on the move, living on the streets or in vans and buses, truck drivers, street performers--anyone who finds themselves in nodes of physical networking like dense urban areas or truck stops with many people passing through.  The details of how to replicate the server are in the section after next, which delves into the code.  Not everyone will need to read this, but everyone should know where to find it, as part of what we will be bartering for as we scale is help from technical experts who can learn the system and replicate the software as well as add to it to evolve it and decentralize the code base. 

The rest of this book describes how to replicate the system, how to use it, how it works, how to develop it into a fully trash-based system, and how we will rebuild mathematics to support this venture.  

\subsection{The Book of Geometron}

This book itself is organic media. It is intended to teach its contents to a reader(or rather a small subset of readers) to the level where they can then teach another.  This should enable them to re-write future improved versions.  In this section I describe how the book was put together, where the files are stored, how to edit them and use the \LaTeX document preparation system to make the files required to produce a finished book.  This means you also need to know what is required to get the physical book printed at an on demand printer, get all the metadata required for publication and distribution, and sell your version in retailers both large and small and online and off.  Thus even the book is fully decentralized in principle: if it costs you nothing to set it up to sell, you can sell only a half dozen copies and it will be a net positive, and then if the next person does this it will also be positive and so on.

If the book is decentralized in this way of distribution it has many advantages.  If the book turns out to be disruptive enough that people try to use lawsuits to shut it down or harass an author, but there are 10's of thousands of new authors popping up all the time, it will be impossible to shut down. As some versions turn out to be dangerous or illegal, other versions can immediately be published with omit the offending content. Also, many editions will mean some will get much better than this initial manuscript.  Decentralization means that as the manuscript finds its way into communities that speak different languages, the translation can happen without any centralized effort. So for instance if someone translates from the original English into say French, and then it spreads around in areas bilingual with French and some other language like Swahili it can go directly from the French to the Swahili without any involvement of the initial English speaking writers.  By avoiding copyright, these improved and translated versions can then get translated back into English and sold yet again under yet anotehr edition.  Having editions be unique can create a market for unusual editions, further pumping money into the system and stimulating further development of the book.  I would rather see 10,000 people make 100 dollars each selling their own editions of this book to just their friends than see me as the initial author make 1 million dollars on 500,000 copies of one edition.

It is not my intent in the long run to make money on Geometron. It is my intent to create a network which allows us to live without money by directly bartering what we need to survive(food, a place to sleep and work, medicine, transport) without use of money or any production in the old consumer economy.

All editions are published with a public domain license for everything but the final pdf.  The final pdf is published under the minimal copyright required for an author to create the needed publication metadata to get distribution outside of the on demand press used for printing.

Each chapter is a .tex file, using standard \LaTeX.

This work must replicate itself completely.  We show here how to edit each chapter, publish them to a public Github repository with detailed instructions for further replication, compile the document to a .pdf in book format, and self-publish the fully compiled book on Lulu Press. We then guide the reader to follow the instructions on Lulu to get all the needed copyright metadata for official distribution through normal publication channels.  We then describe how to order just a few copies, sell them along with other parts of the system here at a markup, and use the profits to buy more print copies of their own book to place in bookstores and libraries as a fully guerilla activity with no official sanction.  This is a little twist on the methodology of Abbie Hoffman's ``Steal This Book.''  In Steel This Book, book sellers had to buy the book, which readers inevitably stole, cutting into book store profits.  We use guerilla production methods to distribute it into bookstores without them spending money.  They are faced with a choice: go along with our program and take free money from customers for the book or make trouble for us, throw the book out, and loose what is for them totally free money.  If they sell the books for a profit, it benefits our network, because it spreads our ideas but also because it creates a value stream in the existing economy based on what we create, which gives us power in that network.   Bookstores want money.  But we want network centrality and the ability to control how information flows in a network, and free distribution shifts the power to us.  This is why the self-replication of the book is so important.  If you want to make your own spin on this book and make it more of a best seller, you do it, but if you leave it open, you hope someone else does it again, and that it keeps getting better as it replicates.  