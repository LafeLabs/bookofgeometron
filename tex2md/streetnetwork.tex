
\subsection{What is the Street Network?}

The consumer society we live in today relies on a vast and well-organized collection of networks to function.  There are the networks which channel freight from one place to another, like trains, trucks and boats.  There are the human transport networks like planes, personal cars and buses. There are the invisible networks of power based on who when to school with whom, family connections, or business connections.  And of course there is the global information network centered on the Internet which connects everyone together by information.  

What we seek to build is the means of replication of trash-based technology in order to propagate our new civilization built entirely from trash.  This new system will be much more localized than the existing one.  Rather than needing a constant high rate of movement of very large quantities of physical goods, we plan to build systems which will ultimately use material directly available in our physical environment rather than from far away.  Even in places with limited agricultural capacity, I believe that with the superior technology which we can build using a more organic system that we can build dense food production and water purification everywhere once we put our minds to it, making almost all large scale movement of goods un-needed.  We will still move goods and people, but more by choice than necessity for personal reasons.  When globalization does not force everyplace to be identical, travel can become an adventure again!

Part of the ideology of the existing Internet-based culture is that place no longer matters.  People appear on a teleconference or send an email and no one cares where they are. People arrive for a meeting in a business hotel or conference center and it is identical to every other place in the world.  The ideological basis of the Geometron Street Network is that place \emph{does} matter and \emph{should}matter.  We reject the idea of ``nowhere''.  Everyone is somewhere.  Whether you are in a rural area, a suburb, a big city or a highway rest stop you are always \emph{somewhere}.  Furthermore, that place you are has its own local geographical logic. It has its own crossroads, its own nodes of power and connection, whether it is the local pub or coffee shop or the exit of a subway station.  All these little places on the scale of one human body have real meaning for the people who inhabit that space.

We want an information network based around physical replication of technology from trash.  To stimulate the replication of the Network, we need it to create value for people who use it and operate it. This value can be of many kinds: it can directly provide physical goods people need, it can facilitate business in the monetary economy, it can provide mutual aid to a community, it can create local social connections, can build network power for users, and any of these values can be traded for materials and space needed to continue to expand the network.  

The Street Network consists of the people going out and spreading all this, the web servers we use to do it, and globally visible web pages which serve as links to connect users to us and our network.  

We buy domain names which are linked to a place but not property.  We avoid .com  and focus on .org, .net or .xyz domains.  We avoid specific addresses or names of companies.  We choose names that describe shared resources which are public enough that no one is in the position to claim ownership of the name.  This can be the name of a neighborhood or street, a body of water, a park, or a mobile shared resource like a mutual aid bus used by people who already do not use property.  In the physical location described by the domain we create and share physical media which points to the domain.  In its simplest form this can just be a hand written cardboard sign.  However, we can also use Geometron to make various self-replicating physical media which transmit the domain, such as laser cut spray stencils and the self-replicating clay tokens described later in this work.   We also can fly a black flag of the place with cut out colored felt sewn on a black square of cloth, which is described in the Scrolls distributed with the system. 

Each domain is hosted on some commercial server for the time being, from providers such as Dreamhost or Bluehost.  Free web hosting can also be used at 000webhost.com and is recommended if you have no money.  The Geometron software can then be installed on each of these servers according to the instructions in the next section(and repeated on each Geoemtron instance in the Scrolls), making it just another identical instance of the software to what exists on all the local Raspberry Pi based servers.   The primary purpose of these globally visible domains is to point back to the local server.  In its simplest form this is just a description of when and where you might find the server.  Photographs of the network Operator and their gear can validate the system to passerby.  The Map format described in a later chapter can also be used to precisely identify physical places where infrastructure can be found.

The deceptively simple structure described above is the Street Network.  We are adding digital media technology to the oldest network in the world: the physical paths of movement.  We will use this to follow all those paths, from  superhighways to ancient footpaths to natural harbors, just as other ideas have traveled throughout human history.  If we find the most powerful nodes of geography we can build a network of staggering power with a relatively small number of people initially.


\subsection{Users}

The uses of this network are very different for different people, just as is the case for the existing global networks like the Internet. In this section I will discuss some of the different user groups and how our network can provide value for them.

\textbf{Operators.} We are the start of this.  Get a Raspeberry Pi and install the system. Get a domain, install the system and point to your server.  Go forth and share! Ultimately those of us who build and share this system will form a very powerful network of mendicants.  The mendicant tradition has appeared many times in many places in history.  A mendicant is someone who is totally devoted to their faith(they are generally religious orders) who renounces wealth and travels with no possessions asking passerby for donations to support them.  This has traditionally created contradictions, as these orders have a way of gaining power and becoming anything but poor as they scale up.  As our consumer society has destroyed itself it has driven more and more people into this way of living against their will.  If our network provides vast amounts of value to people we will find that the most marginalized people of today when leveraging the power of this new network can barter for not just survival but to thrive in a new civilization without money, mining or property.  

We follow the laws of Geoemtron listed in the previous chapter as a guide for building this new world.  We teach everyone we meet how this whole system works, and recruit new people as Operators.  Note that the idea of a mendicant order has strong religious overtones, but that this is a completely ecumenical order based on the universal language of geometry.  Geometry has a central place in all existing cosmologies, both ones considered religious and ones considered secular.  The work here presents a way of interacting with the world based on geometry.  In some ways this whole project can be thought of as the start of a free school for teaching a new kind of geometry.  This is a distant descendant then of the geometry schools of the ancient world.  We teach people the whole system; mathematical philosophy, robotics, code, all kinds of industrial fabrication, crafts, fashion, whatever we build we teach and share freely.  

Do not misinterpret this idea of the mendicant as a vow of poverty.  We will be more wealthy than anyone currently living in the consumer society once we scale this Network.   We are building a new world in which no one is poor.  By starting from a baseline of people who have nothing but building better technology than what is presently available in consumer civilization we start by making sure those who have the least have everything: free clean water, free good food, free high technology medicine, free transport, free shelter, free network technology, free air conditioning and heat etc.  If those who have the least have better stuff than the richest people in today's world the world of today will dissolve and be naturally replaced by this new order built from the waste of the old.

\textbf{Traveling kids, hobos, panhandlers, people asking for money or selling things on the street corner.}  A physically local free bulletin board shared by passerby in a high traffic area can allow people asking for money who are currently ignored by passerby as just another anonymous face and cardboard sign a chance to really tell their stories and to share all that they have to share.  When people share their stories they can become part of the emergent physical community of passerby in a location where the network node is located.  When people view others as part of their community they not only are more willing to help, they can have open communication about the best way to help, expanding from just spare change to more comprehensive mutual aid.  Because we clone content from the local terminal to web pages on globally visible domains linked to a physical place, which are advertised everywhere in that place, marginalized people whose only ability to get online is the public library can use the computers there to get the information they need to better survive, and ultimately to thrive and build new communities where they already are.  The way a local network can help people is twofold. First, it is direct, by asking for money and other mutual aid.  But by being physically on location all the time, already with physical media(cardboard signs), people in a given place can aid the network, creating value for the other people in the community who are more resourced, who then no longer view monetary support as ``donation'', but rather as an expense which supports their other business activities.  

In order to see the power of this second means of network support of marginalized people on the street, we have to look more closely at the network nodes we are building.  One of the major types of node is in a business district of a city where there are both homeless people asking for money, on the street all day with physical media, and power brokers who make their living entirely from connections.  These people include venture capitalists, entrepreneurs, lobbyists, consultants, and the rest of what might be called the ``deal-making class''.  An example of this confluence is some of the parks along K-Street in Washington DC.  K Street and adjacent streets is home to a huge homeless population as well as power brokers whose livelihood depends entirely on connections.  If a physical network were built which facilitated direct communication between people along K Street, the people who spend the most time physically on the street can be brokers of information on a network which can be worth a lot to the people who trade in information.  Physically local information networks can leverage the power of physical places with very powerful people walking past all the time who normally never communicate.  Connecting these people up can be dangerous.  But if we provide them with value, it can be worth both a lot of money to them and also potentially something they can barter for giving us space to live and work nearby.  If you facilitate a 10 million dollar deal and the customer knows you can do it again, the least they can do is give you a 100 dollar gift card to the nicest restaurant in the block.  There is no real upper limit on what an enterprising Network Operator could in theory make if they learned to really channel information efficiently in the nodes of global power.  And of course we must remember that when dealing with power brokers their currency is not money.  When the people who currently have the most power in society find themselves dependent on free open networks, those networks themselves will gain power which penetrates that of the existing power structures, potentially creating an existential threat to them.  We must take note of this.

The elements of traveler culture which overlap with ``van life'' are also key to increasing the network effects of the Street Network.  This also links to trucker networks.  People who live their lives on the road can use this network infrastructure to set up complex networks and markets in highway rest stops, Walmart parking lots etc. using either wifi networks in these places or their own hotspots from their phones.  These networks can be of utility to passerby of all kinds, from tourists to truckers to the workers who keep the places running.  Just as existing global social media networks provide value they can charge money for, a physically local network can provide value which people will pay for.  An example use case here is a Street Network Operator agreeing to maintain a backup of and keep posting an advertisement for something a local entrepreneur is trying to sell to truckers.  In exchange for that, they can get directly compensated in gas, right there in the rest stop, without money changing hands.  

\textbf{Food not bombs, street outreach, harm reduction people, mutual aid workers.}  See above.  The people who are working to help the most marginalized members of any given community can better reach that community if there is a physically local media platform where people can share information about resources.  Documents can be posted which explain how to get access to resources, when and where resources will be available, etc.  Because the whole system self-replicates, as with Food Not Bombs, anything which is successful in any given place can be immediately cloned to other nodes on the network.  Food Not Bombs already has a global network of free and open nodes with no property but a very recognizable brand identity and set of behaviors and actions.  FNB nodes are generally already linked by networks both online and via people who travel from one punk house or FNB house to the next.  The whole anarchist network of community houses, FNB's, anarchist infoshops and bookstores, really really free markets, free boxes, etc. can form a basis for a truly free information network carried from house to house and city to city, running on house wifi networks.  

\textbf{Business owners in a shopping center.}  Every business owner has neighbors who are also business owners.  You already have an informal network.  But installing free digital media infrastructure can provide huge value by allowing more mutual aid between neighbors of all kinds, both owners and customers.  Tech giants ignore you.  They demand monetary tribute in order to even have your business listed, and then still refuse to give you an equal footing to the corporate giants which dominate their platforms.  By controlling a local platform in \emph{your} shopping center, you can provide value to customers with articles they write and share with one another which brings them in(just as advertising-supported media has interesting content to get people to look at ads).  And then this medium can have much more than just the ads you would get from a Big Tech platform. You can post really detailed information about everything you do with no restrictions on length, and share across the shopping center.  If you own a karate school next to a dentist, the bored people in the waiting room can read about the history of karate right next to a detailed schedule of your class offerings.  And when parents wait for their kids to get out of karate class they can be reading about clean gums in an article written by the dentist.  Big Tech doesn't care about you.  If you build your own network, you can center it right where it belongs: on the people actually using it, rather than a few oligarchs in San Francisco. 

\textbf{Coffee shop owners.}  Building a network in a coffee shop on the wifi network which requires purchase to use and which has a time limit can create a huge amount of added business for any local business owner.  It also builds community. So coffee shop owners who find themselves with a full shop of laptop drones with headphones on who work for hours, or get kicked out and do the same thing somewhere else can instead find themselves the brokers in a very powerful information network.  Much of the commerce of the world is now code written in coffee shops on laptops.  Creating physically local networks around these already existing groups can create huge power for the users which then benefits the people who set up the infrastructure(again, just like existing centralized social media platforms.)

\textbf{Web developers.}  We need web developers(people who can write HTML and JavaScript code) to be constantly writing more and better software in order to make Geometron a success. Developers who work all day in coffee shops or any other shared space like a co-working space or pub can have a social network based on both co-developing applications useful to all and sharing other resources.  Developers will use the resource of the Street Network terminal/server on the local network in the same basic way as others: they can share their resumes, links to pages of personal projects.  Developers are key to the whole system. We must recruit developers with this book who will rewrite all the code and also the book, replicating the whole system.  The faster our network can get developers into the swarm, the faster the code itself will improve.  Developers are key!!  Developers create servers to share into the network.  I now ask the reader to look up ``steve balmer developers'' on YouTube.

\textbf{Power brokers.} Venture capitalists, financiers, entrepreneurs, deal-makers of all kinds, lobbyists, politicians.  Your network is your power.  Geography matters.  Build a network in the lobby.  Post things on street nodes, build your network, build your power, build your literal street cred. Deal flow. Deal flow the likes of which you have never seen.  Leverage the power of the physical street! 

\textbf{Crafters, makers, jewelers, artists.}  An alternative to Etsy, street vending, or being in a shop.  Post your stuff to the local networks.  This is much more free and long form than existing platforms, you can post images, descriptions, contact info, times and places when you'll be in a place.  This can be way easier than other sales channels for arts and crafts.  You can say when and where you'll be at a place, post a link for contact, and then show up in the network node like a coffee shop to make the physical exchange.  In many cases, because the network is physical and local, there will be barter opportunities as well as direct sales.  A barter economy can develop where people donate materials you use for your crafts as part of how they pay for the finished product.  Removing shipping or transport costs by dealing directly in a physical location removes friction from the market, amplifying dramatically the power of the market, especially for crafts which involve physically bulky objects.  For instance, people can bring in motors and properly prepared plastic sheets and cardboard, as well as rolls and rolls of duct tape, and we can exchange finished products built from these materials and tools, as well as free food, drinks, and supplies, creating a market economy without money as well as without formal business structures(making it easier for marginalized people to participate).

\textbf{Any labor pool of gig economy workers focused on a specific geographic location.}  One of the most obvious of these is the drivers who presently drive for the major rideshare apps who all congregate at the airport to pick passengers up in the same exact place, and yet all of it is currently coordinated via the apps(unless you do the cab line).  The rideshares apps have proven that cities will ignore illegal cabs if they're done at scale.  It would be straightforward for a small team of Network Operators to run a server which replicates to a page which is advertised around, something like a domain of yourairportnamerides.xyz, which tells users how to log onto the wifi network created by an Operator's hotspot near the pickup zone and with a link on the page to the local network address of the server.  All all this IT is doing is directing customers to a dispatcher who manages the drivers over a simple app shared by the collective.  The whole network is run by a team of about 2-4 people.  One person might be a developer, who creates the app to manage all the drivers and post messages from dispatch.  Another person is all marketing, putting up the relevant information in the right places to get seen by travelers but not stopped by the rideshare apps, airport authorities, or the cab companies.  Riders will never have their destination information on the public network, nor will drivers put personal information, but they can work on an open trust model where they are known by dispatch, who has code names for them, and operates a queue app which simply adds drivers as the arrive near the Airport and pushes the most senior driver to the top of the stack, which is passed along to a rider.  Another Operator might be the one who runs the trust network for the drivers, verifying everyone and organizing meetings for the whole cooperative.  This can be used to unionize existing workforces quickly as well, building ad hoc networks which are very hard to suppress visible to everyone on their mobile devices on a local wifi network.  

The same model holds for places where workers congregate looking for short term construction work.  Those locations can have a server where an Operator runs a labor marketplace where a much larger and deeper labor pool can now advertise, but without all having to be in the physical location.  This means a crowd of a dozen workers looking for work can be replaced by an Operator with a sign pointing to the domain where the copy of the market is hosted.  Workers who come by can leave an ad on the local Raspberry Pi Geometron server, and anyone coming by looking for construction labor can just scroll through a now much deeper collection of ads and call whoever they need to hire.  A market place like this can suddenly go from a dozen general laborers to a construction labor market which includes specialists like plumbers and electricians as well as much larger general contractors just looking to save on marketing costs.  A person holding a cardboard sign on a street corner by a giant box home improvement store can now potentially be the broker of an information network on which millions of dollars of commerce flow.  


\subsection{Trash Robot}

The Trash Robot is a self-replicating set of things described later in this work.  It consists of an open brand combined with a way to create self-replicating symbols which represent in principle anything one can express with language.  We use the Geometron system and the Street Network as a vehicle to distribute Trash Robot.    

Trash Robot icon printers form the basis of a symbolic economy.  This means that we construct self-replicating physical media which can have symbols representing anything, and we use these symbols to communicate with each other about how to replicate things. In a numerical economy we exchange money for goods or services based on a numerical evaluation of the ``value'' of those goods or services.  In Trash Robot we use the Geometron language to make constructions of pure geometry which can be used to organize our thoughts and discussions as we share with one another self-replicating technology of all kinds.

Trash robot consists of a collection of methods for building the robots and operating existing machines to make all these icons symbols, as well as the open brand of the fashion and accessories and arts and crafts which symbolize the system.  This brand consists of googly eyes, rainbows, black cotton flannel, cut felt and rainbow duct tape over cardboard and bamboo.  By being a very recognizable brand identity which is generic enough to be impossible to copyright, we make a vehicle for technology which is not property to freely replicate.


