

Dig it up, set it on fire, and bury it.  Our civilization is an ever-accelerating destructive flow of material from mine to landfill.  This work is based on the idea that we can do better. Much better!  We have dug such a vast array of useful minerals up in the last few hundred years and carried out such fantastic transformations on them into useful materials that we now have the opportunity to build a whole new civilization with a completely different structure than the one we presently inhabit.  

Our current industrial system is a tripod resting on these three legs: money, mining and property.  These ideas make up a philosophical framework for understanding and building our world which is failing at an ever-accelerating rate.  In order to build a new civilization, we must study in detail the structure of the existing one.  
  
Money as an idea is so integrated into our world view that it is very hard to even see what it is.  At its philosophical core, the idea of money is that there is a property we call ``value'' which can be denoted by a number.  This is considered so obviously true by defenders of the existing economic system that when challenged, they simply re-state repeatedly the basic ideas immediately assuming that any system not based on money is just the same set of ideas re-arranged. That is, people will argue that doing away with money will simply replace selling a things with a certain ``value'' denote by prices with trading things to which such a value can also be assigned, but just in a much more inconvenient way(barter). Similarly, the defenders of the idea of money will argue that ``storing value'' is an important task, again assuming that some kind of thing called ``value'' can be measured with numbers and that doing this all-important value storing will simply have to be done with vaults of metal if we dispense with money.  

But this monetary way of thinking does not take into account the possibility that the value of a thing might multiply by replication or that value can be created from nothing.  As long as everyone in society is exchanging value along this stream from landfill to mine, we can use the numbers we call ``money'' to roughly create an equivalent to the main type of value, which is physical material.  We can measure how much gold or cobalt or salt we have, and money lets us transform value from one of these physical things to another, easily trading lithium for silver or oil for aluminum.  

Furthermore, of course, there is labor.  In the labor theory of value, again, money is used to assign some fixed value to a certain amount of ``work'' people do to produce a thing. One can superficially dispense with money but as long as we accept that value comes from how many hours an actual worker one does some type of task, we are just shuffling the details around but preserving the structure.  But what of automation?  And what of other forms of even more drastic increases in efficiency which are possible with information technology? Again, once we allow for full automation and many orders of magnitude increases in efficiency, we find that there is no way to create a value system using numbers which actually describes reality as we experience it in an information economy.  Numerous band aids have been proposed, but if we project forward to more and more automation and further increases in efficiency we see that we have to re-evaluate the whole labor theory of ``value'' along with every other theory of value we hold as a truth today.  The problem, again, is that using numbers to denote value simply will never be compatible with the new civilization we must now build in order to survive.

What happens when we create value from something we have an effectively infinite amount of but which we only have a finite need for?  What if, for instance, I live near a landfill with a vast store of plastic and electronic trash, and I want to build a small factory which converts plastic trash into useful furniture for direct use near the material I have.  Suppose the electronics trash has all the material I need to build this fully automated factory, and that the material needed to create so much great furniture that no one near me ever has to buy furniture again, that they can get custom high quality furniture which can be repaired indefinitely without even tapping too deeply into the vast store of plastic trash available directly in our community.  This act of creation brings a thing of great value into being from nothing but information.  In this situation, money fails.  As long as we rely on money to denote and store value, everyone creating value from nothing has to either get someone in the money-creating business to create currency specifically for them or to do that themselves as they create.  

But the more creative our new industrial processes are, the more catastrophically the money system fails.  If I share the information on how to build the thing with the rest of the world, in principle the equivalent of trillions of dollars of value can be created from nothing. 

This failure is not hypothetical.  Creating value from nothing which can self-replicate freely is precisely what software does.  Software produces things of great value with no material input needed at all, and is able to almost instantly replicate to the whole of humanity.  If someone can create almost infinite value with no input of labor, energy or materials(after the initial creation process), what does that imply for the rest of the people?  We are seeing this every day.  Every city in the world right now is witnessing a violent takeover by the replicators.  Those who can replicate their products infinitely for free include the media industry, marketing, software, finance, and all the numerous information based businesses which make up those business ecosystems.  We are finding that at an ever-accelerating rate all the wealth in our society is being transferred from people who produce things that do not replicate, like physical goods or physical labor to the replicators. 

The purpose of the work here is to show that there is another way.  We can create an information based economy made up of self-replicating information which reproduces things of value which cannot be added up using numbers.  This is not some crude replacement for money using barter, but rather a whole new approach to everything: we will have to build a new way of thinking about machines, society, mathematics, and philosophy.  Just as the existing system is based at its deepest level on numbers, we propose basing our entire civilization on geometry.  Geometron is a universal geometric language which we can use to express the self-replicating information we will need to build this new civilization.  

The second leg of our tripod of civilization is property.  As with money, the idea of property is very difficult to examine as it is so deeply woven into the fabric of everything we do.  The world view we are taught in school and at home is that just as everything has a price that everything is owned by someone.  In some cases that ownership might be the state or even some type of ``common property'' but all things are in some way property.   In the dominant ideology of our time, air and water are property, the land is property, the genes in our DNA patented by the drug companies are property, and even these words I am typing are property.  

Again, as with money, we have to examine how this idea is going to fail more and more catastrophically as we evolve into a civilization based on the replication of technology from trash rather than consumption of mined materials using labor.  In many ways the purpose of property is to inhibit replication.  In the case of intellectual property this is in fact its \emph{only} purpose. But even for physical property like land, the whole idea is that if I ``own'' land the real purpose of that ownership is to make sure someone else does not own it. 

The idea of property makes sense when we are all competing for resources we have to dig up out of the ground.  If I dig up 1 ton of silver it means you can't and vice versa.  We are all in competition to be the owner of that silver.  If you are 1000 pounds of silver richer, I'm 1000 pounds of silver poorer.  But we have to see that with an informational economy based on trash as the main input this is no longer the case.  

If I consume a ton of trash from a landfill into useful things, we have to recognize that that ton of trash has a negative value, which changes completely what it means to use it to make things.  If I consume 1000 pounds of oil to make a plastic part, that oil has a cost we now measure in money I have to spend to buy the oil.  But if I consume that from a landfill, the cost is negative!  Regardless of what I make from the plastic, simply having it be something useful at all is of value.  In our existing consumer economy we all actually have to spend money to dispose of waste.  So an economy built entirely on waste streams breaks the whole idea of property up.  If I have a pile of trash on my land and you take it away that has value to me and also to you.  We are no longer in competition in this relationship--the more trash you take the more benefit I get but also the more you get.  

We must also recognize how an economy based on the replication of technology from trash using free information technology changes the incentives in regards to intellectual property.  If I create a new technology now and release it into the world, the only way for me to make money on it is to retain some control, which is now expressed by means of the property system. But if I create a new technology from trash which does something useful, and it's intended to only be used locally, the value proposition changes.  I use local materials to make a thing and directly benefit from it.  When I share it with you, you also do that, but if you then improve upon it and release it back into the network, I can immediately benefit from the improvement.  

If we build feedback loops across a global network we can get exponential speedups of technological development.  What this means is that if I am a technology creator and share my creations freely, the thing I create can be instantly transformed into a thing co-created by a global community which is vastly superior to what I would have made alone.  If my only goal in building technology is to directly convert the trash in my physical environment into useful things, my choices in regards to how I relate to the rest of the world will be based on trying to do that better and better.  If the more I share the more this happens, often with what I make being totally replaced by something much better, my incentives undergo a radical shift.  What I now want more than making a good thing and controlling it is making a thing which entices others to improve it.  Again, as with money, we find that the idea of property inhibits us from doing what we need to do to build this new civilization.  

The third leg of our tripod is mining.  This is perhaps the most fundamental.  The entire basis of our long strange trip from stone tools to bronze to iron to steel to silicon and so on is based on mining.  We need mines to get the materials to make things.   In order to make complex things with many materials, we need a global system which maintains physical control over all those mines using the system of property and the governments which uphold that property.  We also need a global supply chain again based on stability of governments and large institutions to maintain the constant global flow of goods.  In order for our current system to work, every individual element from oil to lithium to uranium must be transported to everyplace on Earth.  Conversely, every piece of land with a resource on it will under the current system be pressured to extract that resource and push it out into the global economy.  

Under the mining regime, every single type of mine is a choke point to the whole global system.  This regime is inherently conservative: any threat to any part of it will make the whole system fail, harming everyone who relies on it which is presently everyone.  In order to keep the system running, therefore, a constant global regime of military force is required.  One cannot have a mining based civilization without military empires to control the large scale flow of materials.  

And again as with money and property we have to examine how the system of mining affects our relations with our fellow people.  As long as value all comes from a mine, we are all in competition at some level for the mines products.  Every so-called ``developing'' nation will be forced by the dominant powers to extract all their resources to benefit someone else since all the nations are in competition to benefit from those resources.  

But when we build everything in our civilization from the trash of the old world this situation completely changes.  My motivation as a producer of technology is now to turn trash into things of value as much as possible.  If you are thousands of miles away, and we never exchange any physical goods, just information, my incentive is now to have you replicate the trash technology as much as possible.  This is for several reasons. First, there is the same reason as stated above in the discussion of property: the network effect.  The more people copy my technology the better it will get, and the more comfortable my own life will become.  But also, if we build a society which abolishes the mine,  as long as other people are mining they will pose a threat to us.  As long as anyone in the world bases their civilization on mining, they will need to build empires and dominate large masses of land in order to keep mining.  So if I want to not be invaded by a mine-based empire it is in my best interest to help every other place on Earth develop the same technology in order to also prevent the violence of the mine from destroying what we build.

In today's world \emph{every} single useful material we need for advanced technology has been pretty evenly distributed to every corner of the globe.  This is totally unprecedented! There is nothing in history even remotely similar to the situation we find ourselves in today.  People do not seem to have really grasped how fundamental and irreversible this is. Even if humanity died out tomorrow and were replaced by evolved crows in 100 million years, the distribution of rare minerals around the globe will remain.  We will never again have to discover from scratch how to find and extract materials like cobalt or tantalum.  We not only have all the materials needed to build an advanced technological civilization from scratch, those materials are already in a very organized form specifically designed to be useful.  Aluminum has already been extracted from bauxite, iron has been smelted into steel, silicon purified into wafers of unimaginable perfection and so on.  

Examining all three of these legs(money, property and mining) it should now be clear that these ways of thinking fall apart when we build all of our technology from trash instead of mined materials.  It is my intent in this work to build a framework for creating this new world.  To do this, we will need to build up a whole world of thought and action from scratch.  The most fundamental focus of this new world is replication.  We study how media replicates, how machines replicate, how software and hardware replicate, how whole systems replicate, and how pure information replicates.  

We also take as an axiom that geometric thinking is more fundamental than numerical.  This is because geometry is what we use to actually make things.  From buildings to microchips to injection molded plastic enclosures, all technology is essentially a geometric construction of one kind or another.  So if we are interested in building media the sole purpose of which is to replicate technology effectively, we find that geometry is the most fundamental form of mathematical thinking.  As with the existing system of thought we currently live in, we will need to delve deeply into our most fundamental assumptions of how the world works.  But now rather than trying to find some abstract truth, as the mathematicians of the early 20th century did, we build up a system of thought based on outcome: that which freely replicates useful things from trash is the goal, whatever that turns out to be.  This will lead to re-evaluating how we think about machines and mathematical philosophy, replacing the theories of ``computers'' with ideas about geometric machines to print symbols.  

One final note to make about this geometric world view.  As with all our mathematics in this new civilization, our goal is replication, not finding some higher truth.  This means that geometry is all based on its meaning to humans.  Even if the meaning is in the angle of a turbine blade which communicates a different level of air movement in an air conditioner, it is this meaning we care about, not some abstract theorem to prove or algorithm to own.  We therefore take language and symbols to be the most fundamental elements of which our Universe is constructed.  We accept that whatever we may think or do, the ``real'' universe is separated from our minds by a veil of language we can never fully see clearly through.  Reductionist science has made the mistake of ignoring this veil and focusing on a hypothetical ``objective reality''.  Whether or not this is a permanent intellectual dead end is of no interest to us here. We want results, fast. We want a better civilization in our lifetimes.  And to do that we build up a new way of thinking about information where our desire to provide direct value to people and replicating that to as many people as possible is our most fundamental axiom.

We call this geometric system of value, this geometric meta-langauge, Geometron.  This is the Book of Geometron, which describes how to replicate the whole system.

To build this world we will first discuss how media needs to change to support this new way of thinking, then how we will physically deploy this media infrastructure to the streets of our world.  We then show how the software works, how it replicates, and how you can add to it, improve it and make it your own to share.  After that, we step through all the different layers of information that make up this new network.  We then talk about our theory of being(ontology), the new underlying mathematical philosophy which we are using to replace the axiomatic set theory that 20th century mathematicians used to describe reality, as well as the theory of machines which replaces the Turing model of computation.  Finally we use this idea of how information works to discuss the self-replicating set known as Trash Robot which we will replicate through the network and also use as a vehicle to help stimulate replication.  

