
A Feed is an ordered list of pieces of information.  This structure of information, in which a set of media elements like images or small bits of text are related to each other by order, which a person can scroll through, is the basis of most social media today.  What we are doing is so radically different from anything you have ever encountered, however, that it is worth taking some time here to explore how exactly these feeds work in consumer media and how differently they work when we move to Geometron.

In a consumer-driven social media system, everyone involved is either a user, a worker for the company who owns the platform, or an advertiser.  In order to interact with the system, a user is required to have a unique identifier based on their user name and password.  Every single thing they do, from how fast they scroll to what they click on to every post they make is tracked and analyzed by the employees of the company. Advertisers then pay the company for the ability to spy on and manipulate those users into consuming more, and it is the job of the company employees to do that as efficiently as possible.  

The flow of information in these consumer systems completely structured around these separate user accounts everyone on the consumer side is forced to use.  Even in supposedly anonymous forums, posts are tied to IP addresses and can't be edited or deleted by other ``users''.  That is to say, in the consumer driven systems, even without names the basic paradigm of the individual user is dominant.  

Each user creates a stream of information which is ``owned'' by them, which is ``their'' feed.  When each user is on the system, however, they see another feed, also called ``their'' feed, which is the sequence of information the company chooses to show them.   This is a mixture of elements from other users' feeds and all the media being forced on them against their will by the advertisers who are paying for the network.  In order for all this to work, the company has to give them just enough elements in their feed that they keep staring at the screen, while feeding them as many corporate manipulation messages as they can get away with to maximize shareholder value and profits.  This has to violate the users' consent or it doesn't work--the whole system has to involve a central power who can control what everyone sees or people would simply turn off the feeds of the predatory manipulators and the system would collapse from lack of financial support.  Also, the existence of ``users'' is needed in the current system again to maximize profits, because that is what allows the manipulators to target based on detailed profiles of who clicks on what and more importantly who will consume what.

Users are encouraged to have things ``go viral'' but always within a controlled environment where the company can shut it down and censor it and where each copy is linked to a user so that again the users' behavior can be carefully tracked in order better to manipulate it later.

Geometron has feeds as well.  We have feeds of images, feeds of text, feeds of symbols, feeds of icons to be printed by robots, feeds of hyperlinks, and various feeds that are complex combinations of these.  Our system of creating feeds is so simple that anyone with a very basic command of web development in HTML and JavaScript can create their own feed applications within an hour or two with copying and pasting of existing apps.  In this chapter we will examine how our feeds work and how they differ from consumer feeds and then talk about the existing feeds we use in this system, largely to connect together other parts of the system.  These feeds are potentially very powerful, however, but it remains to be seen exactly how they get used.  Try stuff!

In Geometron, we do things so differently that it can take some thought to to see just \emph{how} different this network is.  We now recall some of the relevant Laws of Geometron and discuss how they impact what we do on this system.  

\textbf{No property.}  Again, while this sounds radical, it is in fact much more radical than it sounds.  There are no users on this system.  No employees, no company.  No ``core developer team''.  No ``marketing customers''. There is no user data because there are no users.  There are no passwords, there are no logins.  The servers themselves are non-property objects which exist to be shared freely from place to place.  In our system, feeds are just documents. They are always in a pure text format which can be interpreted by the browser and turned into something readable.  

\textbf{Everything is replicates.}  A feed can always be copied and pasted in pure text(using the JSON format). It can be stored in a public pastebin just like scrolls, or stored in a text message or email or embedded in a scroll.  A feed can be copied instantly from one server to another, spreading the whole feed as an object, without any regard for where each element of the feed came from.  Everything we create is designed to replicate.

\textbf{Everything dies.}  Every element of every feed can be deleted instantly by anyone interacting with that feed at any time.  This is how we create a system where replication only happens with the consent of the people.

\textbf{Everything evolves.}  Every feed can be edited by anyone at any time. There are no ``permanent'' files. To share, read, or copy is to be able to edit.

\textbf{Everything is physical.}  The structure described here might sound unimaginably chaotic compared to existing systems where all information is controlled by a central authority and by private user accounts under some type of algorithm.  But that is because we are used to global networks, where distance and geography don't matter. Our system is physical, centered on the community in direct proximity to the physical server.  This means we are only trying to organize information in a way which is useful to the community physically surrounding this server.  This makes the information sorting and organizing vastly simpler than on a global network.  This limitation in the who and where parts of our system is important partly because it is what makes our model work in which everyone can edit anything at any time.  This only works if people don't edit the same exact piece of information at the same exact time.  Because we are all effectively in the same physical room, but are assumed to be wandering in and out, this is a dynamic which can be built up socially in an organic way.

\textbf{Everything is fractal.}  Again, this is how we can make sense of the potentially overwhelming complexity of a feed without users. As soon as a feed becomes too complicated, we just fork it into multiple feeds based on sub-topics.  Part of the chaos we are managing here is the potential collision of edits referenced above.  Again, this can be resolved with a fork.  The program fork.html will create any number of sub-pages below the top level, where as we add more fractal divisions we can avoid edit collisions.
   
\textbf{No money.}  We do not have ``customers'' who can simply buy their way into forcing people to see their information.  While network Operators might take donations for teaching people the system, because everything people don't like is deleted, there is no way to enforce the contract structure used in consumer society to pay for advertisements.  Money plays a role in our system initially as we all need to survive somehow, but is not the backbone of the system like on a consumer system.  Also, consumer systems require constant money flowing in in order to keep the server farms running and all their employees doing work they hate to manipulate people.  Our servers are passed around through a community in ways of direct use to that community, so no global cash flow is required to keep them running in the long run.

While I am trying to keep this discussion non-technical, it is necessary to say something about the format used for Feeds in Geometron.  Geometron feeds come in two varieties: a directory with files in it, and a JSON file.  JSON is a format which stands for JavaScript Object Notation, and which is a very simple and universally recognized text-based format to organize information.  Any of the hundreds of programming languages in common use today will have already built in routines to handle JSON, and any programmer on any system should already be familiar with it.  Like Markdown for the Scrolls, we select JSON because it is easy to copy and paste, and is as ``lightweight'' as possible, meaning we need very little added information in the form of weird looking symbols and so on to store things.  

The most basic type of JSON structure we use is the array, which is just a collection of things separated by commas, inside square brackets.  The other main thing that exists in JSON is the ``object'', which is a collection of pairs where one element of the pair is a piece of information and the other is a name for that information.  This idea of creating abstract objects which map names of things to things and organize them in this way is part of the ``object oriented'' idea which is the basis of most modern computer systems.  Never be afraid to edit and read the raw JSON!  You don't have to interact with it but you should not be afraid of it.  If you destroy a JSON file, the file you started with can just be replicated again and again to avoid the fault.

The first feed we will discuss here is Chaos Feed, which is perhaps the most basic, and is really just designed to show the concept.  Chaos Feed can be found along with the other feeds from the Feed Scroll which should be linked from your Home Scroll on whatever Geometron system you are using.  To post, type and hit return. That's all.  To delete, hit one of the red ``X'''s.  If you are using a keyboard, the up and down arrows scroll through the existing elements.  To clear and start over, click the button in the upper right.  To reload click the icon in the upper left.  To see and copy the current feed, navigate to data/chaosfeed.txt on any Geometron system and you will see the JSON for that feed.  To copy a feed from a remote pastebin, use copy.php as with the scrolls described in the previous chapter.  To edit a feed or to edit the application chaosfeed.html, use the code editor editor.php.  From editor.php you can edit any code on the system, including both the file chaosfeed.html and data/chaosfeed.txt.  While most people will probably not want to edit this code, it is simple to edit for people who know basic web development, and so you can ask around and find people who like to do that to evolve the system.  I do not know how people will use this. Try stuff!

Another feed almost identical to Chaos Feed but with different applications is urlfeed.html.  This is the same interface: you put something in the input and hit return and it posts.  But this is for links. Each entry turns into a live HTML hyperlink which actually links to whatever the link points to.  To use this, find a link either locally on your system or globally and just copy and paste it in and hit return.  Edit, delete and share just as in Chaos Feed.  This feed unleashes the power of Hypertext, the central technology of the World Wide Web.  Connect from your humble local feed to the whole world! Connect your system together. Connect everything.

Yet another feed of almost the same format is the Global Image Feed, or globalimagefeed.html on your server.  In this, you again just paste links in and they post, but in this case you are pasting image links which will then appear in the Feed. Again, these can then be deleted.  Click on any image to see the url for that image appear in the text area.  This Feed links up with the alignment system for tracing Icons which will be dealt with later in this book.  But it can be used for anything, including just sharing random images from around the Web or use as inputs for any other Geomtron application.  We will deal later with a number of such applications which can call on this feed and use it to load useful images to do things with.  Once again, delete anything at any time you want by clicking a red ``X''.

I said above that there are two types of Feed, and we now turn our attention to the second one: files in a directory.  We use this for images uploaded to a server as well as symbols created on a server.  Images can be uploaded to a Raspberry Pi Geometron Server using localimagefeed.html.  This is also used as part of the workflow in several other part of the Geometron system.  Note that there is a maximum size on images, so it can be useful to screen shot images on a mobile device, then crop the screen shot to get something closer to 1 megabyte or smaller before uploading.  The maximum is about 2 megabytes.  This is to keep us from having exploding sizes of data based on very high density images.  This is a key feature of our physically local social media system!  We can use this to photograph objects and places in direct proximity to the server and upload them and use them in order to build media which directly represents our physical environment.  To upload, click ``choose file'' first, and select a file to upload from your device, then when that file is selected use ``upload image'' to upload the image.  To delete, use the red ``X''.  All you are seeing here is a list of the files in the directory uploadimages/, which you can manually examine and interact with as well.  This can be an easy way to transmit images and memes in a peer to peer way, as a meme or image can be uploaded from a mobile device to a local server, which another user can then download to their mobile device, allowing for rapid peer to peer transmission locally over the wifi network.  As with the other Feeds here, this is deceptively simple, and can by itself form the basis of a whole new type of local social media networking.

The real heart of the Geometron system is the geometric programming language presented later in this book.  Among other things this creates image files which are symbols in either the vector graphics format .svg or the bitmap format .png.  These are all stored in the directory symbolfeed/.  They are viewed and deleted using the feed program symbolfeed.html. This whole system will be explained in detail later in this book, but for now it is just important to know it is there: this is geometric social media, in which you create geometry and share it freely.  Again this is deceptively simple, and can form the basis of a very wide range of technological activities which will be described later(along with the Icon Feed, stored at iconfeed.html).

The Feeds described here are programs which run on a Geoemtron system.  We must also consider Feeds in a more abstract sense which we will use to build up our whole system.  Ultimately the whole system will be based on the Trash Feed, the vast global feed from mine to landfill which we will redirect toward our new civilization. This feed is made up of many small local feeds, which have a fractal structure.  For example, a trash bin near a kiosk in a park selling bottled water will be a constant source of plastic bottles.  This type of feed is part of our system in a general sense even though it is not software.  But it is not totally separated from our software either.  Our system includes robotics which can print icons into plastic bottle caps with a heated tool, and those icons are shared on the Icon Feed, so there is a crossover between the physical feeds of trash-sourced objects and this less tangible software system.


