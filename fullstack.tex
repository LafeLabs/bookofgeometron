\section{Full Stack Geometron}
\begin{itemize}
\tightlist
\item
end goals
\item
hybrid fabrication technology
\item
clockless operation, hardware GVM
\item
image stack, hardware map processing
\item
roctal, storage hardware, scaling, read/write/operate
\item
large projections in Skeletron booths, Geometron station, fully immersive VR/AR at 2 meter scale
\end{itemize}

\subsection{Goals}

Our end goal is an information technology which has zero mined materials.  Every material input in the creation of any technology in our system should come from either a waste stream or from the living world in a closed loop(e.g. naturally harvested sticks and tree sap, goose guano). 

Our initial goal has been to rely on the most open and lowest cost off the shelf hardware possible, which right now is the Raspberry Pi.  As the system expands we will want more and more ways to fully wipe hard drives and sim cards and replace them with Raspberry-Pi-like systems running some form of Linux, with web servers installed with PHP running the same sofwtware. That will be the next step after the Pi phase.  But after that we will try to break up the system into modular parts which can be scavenged from more and more destroyed electronics.  

We see this as a continuous transition where in the beginning we are just running an app on an existing phone, then we wipe the phones and run our own OS, and then we start taking screens, batteries, and other components out and mixing and matching them.  It is in this mixing and matching where the next phase of development will happen.

As this mix-and-match upcycled Linux system evolves, the Trash Robot will also evolve more elaborate ways to fabricate electrical connections at the millimeter scale.  This will eventually lead to more and more complex circuits being built up from scratch, connecting various upcycled components at the individual level(a single resistor, capacitor or MOSFET for instance).  When these circuits can be fully integrated in three dimensions into collections of upcycled components, and when we have some degree of automation in this process, we can start to say we have a fully upcycled system.


The Trash Robot technology can, when it is expanded, be the basis of a milimeter scale interconnect fabrication technology.  Circuits can be immersed in an electrolyte solution and an electrochemical probe controlled with the Trash Robot can 