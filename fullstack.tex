\section{Full Stack Geometron}
\begin{itemize}
\tightlist
\item
end goals
\item
hybrid fabrication technology
\item
clockless operation, hardware GVM
\item
image stack, hardware map processing
\item
roctal, storage hardware, scaling, read/write/operate
\item
large projections in Skeletron booths, Geometron station, fully immersive VR/AR at 2 meter scale
\end{itemize}

\subsection{Goals}

Our end goal is an information technology which has zero mined materials.  Every material input in the creation of any technology in our system should come from either a waste stream or from the living world in a closed loop(e.g. naturally harvested sticks and tree sap, goose guano). 

Our initial goal has been to rely on the most open and lowest cost off the shelf hardware possible, which right now is the Raspberry Pi.  As the system expands we will want more and more ways to fully wipe hard drives and sim cards and replace them with Raspberry-Pi-like systems running some form of Linux, with web servers installed with PHP running the same sofwtware. That will be the next step after the Pi phase.  But after that we will try to break up the system into modular parts which can be scavenged from more and more destroyed electronics.  

We see this as a continuous transition where in the beginning we are just running an app on an existing phone, then we wipe the phones and run our own OS, and then we start taking screens, batteries, and other components out and mixing and matching them.  It is in this mixing and matching where the next phase of development will happen.

As this mix-and-match upcycled Linux system evolves, the Trash Robot will in parallel evolve more elaborate ways to fabricate electrical connections at the millimeter scale.  This will eventually lead to more and more complex circuits being built up from scratch, connecting various upcycled components at the individual level(a single resistor, capacitor or MOSFET for instance).  When these circuits can be fully integrated in three dimensions into collections of upcycled components, and when we have some degree of automation in this process, we can start to say we have a fully upcycled system, but it will still be the same software stack, with Geometron running on PHP, Apache, and Linux.  

This hybrid interconnect technology can be based on immersion of circuits in an electrolyte solution with a robotic probe which applies current to deposit metal, as well as techniques where the robot probe is modifying clay, and a series of clay fabrication steps are carried out in layers just as conventional micro-lithography adds layers.

In parallel to all of this we need to be developing battery technology using aluminum from beverage cans with carbon from charcoal made from organic material like burned dung.  We also need energy production based on ultra small scale generation using found materials. So our goal will be to find a reliable waste stream like fans from broken computers, turn that into a generator of electricity which can run on very small scale water or wind at the level of from 100's of watts down to single digits of watts.  Using the aluminum air cycle with organic waste and beer cans from the trash it should be possible to have a zero-mine closed loop energy production and storage system at the level of IT platforms.  Building IT infrastructure as a shared resource in some local public place which stays fixed in a location removes the severe demands currently placed on battery technology for networked devices which have to fit in a pocket.  If servers also serve as the information terminal, with a giant screen for all passerby to see, a battery could be a 6 foot high stack and it would have no real impact on usability.  This relaxed restriction combined with using waste aluminum cans and locally sourced burned dung should make aluminum air cycle batteries practical in a way they have not been traditionally.

Another technology which has to be developed as part of the Trash Robot road map is printing in permanent materials at the micron or sub micron scale for long term information storage.  This has to be developed in parallel with readout technology which uses upcycled camera chips from trashed smart phones as sensors in our hybrid fabrication technology which can read the information back out.

Developing our own information storage system is also a part of a much more radical shift in information technology, where we abandon the whole operating system and build hardware for direct Geometron-based document processing.  To do this, we re-adjust what a machine does.  The only goal of our machine is to freely share documents over a network.  When a machine is not operating something mechanical like the Trash Robot, all it does is display pixels on the screen. Those pixels only change if a user engages with the system. We therefore totally replace the model of a Turing machine doing arithmetic on ones and zeros with a model which just uses the Geometron Virtual Machine to draw vector graphics in the screen pixels(this can include all text in all human languages, which can all be drawn using Geometron fonts) and a system for displaying bitmaps in geometric positions as described in the Geometron Map format.  To do this, we will need an image stack, a memory system which records an array of bitmaps of images, and a Geometron Virtual Machine.  We will also need a network of switches, where user interaction with keyboards, pointers, buttons etc. can trigger behavior like scrolling or editing a scroll, zooming on a map, drawing a symbol, or clicking on a link which goes to some other file.  

All this does not need a constantly running clock. This is one of the biggest differences between traditional computers and a full stack Geometron machine.  When there is no user interaction it simply does nothing. It doesn't just have minimal operating system services running in the background, it doesn't even have a clock. No aspect of the system changes state in any way. It is just a collection of pixels getting enough power to stay in whatever state they are in, until the user engages, and then it only changes state in direct response to each user interaction.

The data we store in the system can be either short term or long term.  Short term data can be stored in a modularized memory unit based on easily found memory chips from trashed phones, but with standardized interconnects added by the Trash Robot fab technology so that they can be easily plugged in and replaced after they are upcycled into our system.  Long term memory is designed to be \emph{very} long term.  We can use scaled down versions of the Trash Robot clay printer presented here to print in clay at a much smaller scale.  As with the larger clay symbol tokens, this can also be self-replicating, as the depressions in the clay can again be used to stamp out inverse images which can be used to make more.  So with no automation or IT at all someone with no tools or even electricity should be able to clone data storage media. 

The format for long term storage is in arrays of squares or dots which are either in a 1 or zero state, filled in or empty.  If possible, the data all have redundancy, with Arabic numerals written along with each Geometron bytecode byte, as well as in some cases an actual print of the symbol being represented, and alignment marks for that individual byte.  This redundancy and alignment mark system should make the data maximally robust against decay and future technological limitations, as if it were absolutely necessary, the 