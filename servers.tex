
Servers are machines that store and share documents in the Geometron system.  There are three kinds of servers we work with: the Raspberry Pi servers that form the real backbone of the network, globally visible hosted domains used to point people to local Pi servers, and developer servers for editing and sharing new versions of the Geometron software itself.  

To spread the Network, the most fundamental form of replication is replicating the local Raspberry Pi server.  To do this, we want to either buy or barter for the parts, install the software, and teach someone how to operate it so that it can be sent out onto the Street for public use.  Given the choice, we will always barter for the parts.  If we can find people who support our cause and have extra technology hardware like old keyboards, screens, or even Raspberry Pi boards, it will be easiest to take direct donation of hardware on location by an existing server than to deal with purchasing and shipping new hardware.

The elements of the basic Raspberry Pi server are:

\begin{itemize}

\item The Raspberry Pi board itself.  This is a circuit board about the size of a deck of cards.  All of them should work! Older ones might be slower but they should all work.  This has all been tested on the Pi 3 and Pi 4.  Boards are generally between 40 and 50 dollars, but again if you can barter them that's ideal, as they are often sitting idle in peoples desk drawers.
\item micro SD card and SD card reader to write the card.  
\item USB keyboard.  If possible, find the one without the number pad so that it fits more easily in a backpack, this can make a huge difference in portability.
\item USB mouse.
\item HDMI Display. Not all work, but all can be made to work. Ideally you want a small screen which can run off the same battery as the Pi.  If you are setting up in a fixed location you can use any standard modern TV screen, which good both for needing fewer new resources and for visibility.  A large screen display for a server can be a great way to have shared physical social media, where people can read documents on the big screen without needing any device of their own.  For small portable ones we recommend buying ones specifically sold for the Pi, and which say that they don't need any special installation of software to work.
\item HDMI or HDMI mini cable.  There are some tiny Pi displays which don't need this because they connect to the pins on the board.  Note also that you will need the HDMI mini cable for the Pi model 4 but the regular HDMI for all other models.
\item Lithium ion polymer battery packs. This is only for portable installations, but it is important to have something modular and portable with a lot of power storage capacity.  We recommend the TalentCell batteries as being easy to charge, easy to carry, and having both 12 V and USB output.  They also sell solar chargers for those batteries, which are useful to have for long stretches of being away from power.
\item Wall power.  For the Pi model 3 and below this means a wall supply with the same USB micro and for the model 4 it is USB C.  You will want a wall supply that can put out 3 amps at 5 volts.  Also, if you are running everything of 12 volts you will already have a wall supply that came with the battery packs listed above.
\item Wifi hotspot.  This can just be a phone with the hotspot turned on.  But it can also be a mobile hotspot from the phone company which has its own wifi and connects to the network.

\end{itemize}
 
When you have all the materials to make a server, you will want to start by setting up the Raspberry Pi in the normal way documented on the Raspberry Pi website.  Follow the instructions on there to copy the NOOBS operating system onto the SD card.  You can also buy SD cards with NOOBS already installed.  The Raspberry Pi home page is www.raspberrypi.org.  Buying Raspberry Pi stuff can be done in person at some electronics retailers, some maker spaces, and online from numerous retailers, just search and look around.  Sunfounder is a great source of compact portable screens for the Pi as well as other Pi things.

Note that part of setting up the Pi is logging onto some kind of wifi network, which is similar to on any other computer system, you click on the wifi icon in the upper right of the screen, select a network and put in the key.  If we are using a hot spot we will want to select a simple name and password like using ``geometron'' for both, and post as widely as possible to potential users what that is so it's easy for them to remember and log on.  In order for this network to be visible, all users must share a wifi network.  It is also possible to point a global domain name to a local Pi visibible to the outside world, but that is beyond the scope of this book.

Once the basic operating system is set up(set it up with no password) you will want to install the web server, the PHP language, and the Geometron server.  To do this you can follow the instructions at github.com/lafelabs/thing, which is where all the code for this project lives and from which it will all be copied when you install.  This copying process copies all the detailed instructions to copy the system, so if you find any instance of the Geometron system on a global or local server you can follow the instructions on there to replicate.  

When the Geometron server is installed on the Pi you can interact with it by opening the web browser on the Pi and pointing it to http://localhost.  This should now look like any other server in Geometron.  This can be used to create, edit and replicate documents of all the formats in the Geometron system, which are documented in the next several chapters of this book, as well as on each instance of the system.

When you set up a new Pi server, you will want to copy the Pi scroll to the home scroll, and there is a link to do that in the default screen.  When the Pi has a correct Home Scroll there will be a link to open a page which makes a QR code for the server.  You can then scan the QR code on the screen to log on with any mobile device which is logged onto the same wifi network as the Pi.

The second type of Geometron server is on remote hosted domains.  As discussed in earlier chapters, we will buy domain names based on generic places that are not owned by anyone but have a physicality of some kind.  For example streets, parks, rivers, neighborhoods, truck stops, or mobile mutual aid stations.  To install the Geometron server on a globally hosted domain, just copy the file replicator.php from any existing Geometron server then point a browser to it.  This is documented in the README documents.

Finally the developer server is used for local editing of Geometron on a private computer which can then be pushed to public repositories in a host like Github.  This is done using the built in web server of the PHP language.  If you are using a Mac, PHP is built in and you can run all this from the command line.  If you are on a PC you will need to install the Ubuntu machine under Windows 10, install PHP and use that command line.  In either case, start a new Github repository, set up whatever you would normally use for development via Github and put the file replicator.php in there.  Run it with php replicator.php at the command line.  Then while in that directory, run php -S localhost:80 and point a browser on that machine to http://localhost.  Now you can operate Geometron as normal.

To edit all the code on the system, use editor.php, which is linked from the README file.  To point the next replicator to your new instance of Geometron, edit the code in php/replicator.txt to point to the /data/dna.txt file of your new instance, and then convert that to php with text2php.php, which is linked from the code editor.  

The details of the system should be documented on the system itself and on other linked media, so it is redundant and tedious to delve too deeply into technical details here.  This is just here for completeness so that the system is described in broad strokes and so there are pointers to all the bits you need to learn about if you want to move from just using the system to building your own new systems based on it.

These three types of server are the whole system!  There is no company selling server space or running a central code base.  Each individual server of any kind has the whole system on it. If every system on the planet were deleted in an instant, you could repopulate the entire world with the one copy on the individual Raspberry Pi you are using, or the Github repository you cloned it from or the web page of the local street that connected you to the Network.  Github repositories replicate code, hosted domains point to Pi servers, and Pi servers are the medium on which all documents can be created, replicated, and shared freely across the whole rest of the Network.  All this can happen with no company, no organization, no centralized code base, no centralized brand or naming convention, no authority, no property, no cash flow, and no presence in any app store. And all of it is open, clear, self-documenting, and easy to copy.

Now, go forth and multiply!  Let us first make a million of these with the Raspberry Pi, then learn to make them on old hardware which we install stripped down Linux systems on, and then finally on Geometron hardware built entirely from trash using the full stack Geometron system described later in this book.  Millions of servers can serve hundreds of millions of people.  When we scale to using all reclaimed trash for hardware, we can scale to billions of servers, eliminating the personal layer of networking completely, with ubiquitous open and free media shared in physically public spaces.  This technology is clearly already possible to build if we choose to do so, and it can clearly be done for free given the very high rate at which the existing consumption-based system is pumping out electronic trash with all the elements of media(screens, batteries, radio transmitters etc).



