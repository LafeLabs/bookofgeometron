\url{index.html}

\section{Organic Media}\label{organic-media}

\begin{itemize}
\item
  everything replicates
\item
  everything evolves
\item
  everything dies
\item
  no money
\item
  no mining
\item
  no property
\end{itemize}

Organic media exist to replicate that which we desire to replicate.  Its purpose is to facility replication of things.  It contains both information on how to replicate things and information required to inspire people to \emph{desire} to carry out replication.  We must give people a reason to replicate as well as a means to replicate.  Thus one form of Organic Media is instructions on how to build something you can sell.  To maximize replication we are less concerned with how much money someone can make via replication than we are with how little they have to spend to initiate replication.  We want to minimize the cost of each item a person might buy to sell, the minimum number required to be purchased, and also minimize the amount of labor and skill required to make the thing.  The ArtBox is an example: each box contains the content to make another box.  Each box is awesome and can be sold.  The ArtBox itself is Organic media.  a bunch of SRS gets moved here.



path of replication: how to install and how to teach others to install,
how to replicate the code, replicate specific files, how it's all
structured, editing the code itself

the structure of sign pointing to domain which points to terminal which
is near a place which is labelled by domain(this needs to get explained
early since it will be the referenced in everything, as it's basic to
the structure of the System)

Editing: how files work, how they don't work. No databases. no private
data. No file that cannot be edited.

Deletion: cheap sd cards can be wiped any time. All files can be deleted
or written over by anyone at any time. Domains are chosen to be of no
real value, are disposed of as soon as burned, never sold. Evaluate
entropy of domain names to prove we can always have zero value of names
even as our network grows exponentially. ``Permanent'' deletion less
critical in a world without private or personal data.

Cybersecurity: there is none. Security is for private property, and
there simply is none on this network. This network must always be
disconnected from the property-based Internet or private systems. One
way paths of information for protection.

draw from the SRS paper and the existing organic media paper(but the
formal SRS idea goes in ontology chapter)

File types independent of technology: symbols, feeds, maps, scrolls.
These can be described without any specific implementation of
\emph{either} software \emph{or} hardware. Specific implementations will
be dealt with in their own sections.

How files are structured, overview of file types, what they do. All of
this is as a generalized specification which users can use to rewrite
from scratch quickly.

Example of physical organic media: The Art Box, replicator card,
replicator shape, object itself, scroll

We will return to the Art Box again and again as we go through the
system, illustrating how it can replicate in the Geometron system and
how it fits into all the parts: symbols in ``symbols'' for the shapes
and net, the set in ontology, maps, feeds, scrolls, scroll replicator,
role in magic as one of the Trash Robot avatar objects

The user page, basic operation of Geometron as it exists today, just
viewing now editing, but with pointers to editors, jump on and look at a
file RIGHT NOW, on one of my pages.

In organic media chapter we introduce geometron and say what we're going
to be doing in this book. What we are doing is providing a
\emph{specification} and an \emph{example} of an instance of that spec.
We will use the Pi, and remote hosted pages. But expect this spec to be
used by many coders on many platforms in many ways if successful.

This is the chapter which defines Geometron, states the goal here.  We want an information network which facilitates the free replication of technology built from trash.  To do that, we will need text documents, hypertext graphical documents(like slides), feeds of information elements, and generalized symbols and symbolic languages which can be used to create vector graphics and control machines for automation.  Creating this language and a hardware platform to transmit it, which replicates itself will be a seed of a new economic system.  In order for this to self-replicate we need the media to create value for its users.  The next section describes the physical network which our system runs on.   

In this work, we describe a system called Trash Robot and a language and IT platform called Geometron which together can serve as a platform for a post-scarcity economy without money.  We will describe how the symbolic language can be used to create automation technology from scavenged trash components, and demonstrate with a practical machine which itself is used to fabricate materials which can then be sold.  

