\url{index.html}

\section{Organic Media}\label{organic-media}

\begin{itemize}
\item
  everything replicates
\item
  everything evolves
\item
  everything dies
\item
  no money
\item
  no mining
\item
  no property
\end{itemize}

path of replication: how to install and how to teach others to install,
how to replicate the code, replicate specific files, how it's all
structured, editing the code itself

the structure of sign pointing to domain which points to terminal which
is near a place which is labelled by domain(this needs to get explained
early since it will be the referenced in everything, as it's basic to
the structure of the System)

Editing: how files work, how they don't work. No databases. no private
data. No file that cannot be edited.

Deletion: cheap sd cards can be wiped any time. All files can be deleted
or written over by anyone at any time. Domains are chosen to be of no
real value, are disposed of as soon as burned, never sold. Evaluate
entropy of domain names to prove we can always have zero value of names
even as our network grows exponentially. ``Permanent'' deletion less
critical in a world without private or personal data.

Cybersecurity: there is none. Security is for private property, and
there simply is none on this network. This network must always be
disconnected from the property-based Internet or private systems. One
way paths of information for protection.

draw from the SRS paper and the existing organic media paper(but the
formal SRS idea goes in ontology chapter)

File types independent of technology: symbols, feeds, maps, scrolls.
These can be described without any specific implementation of
\emph{either} software \emph{or} hardware. Specific implementations will
be dealt with in their own sections.

How files are structured, overview of file types, what they do. All of
this is as a generalized specification which users can use to rewrite
from scratch quickly.

Example of physical organic media: The Art Box, replicator card,
replicator shape, object itself, scroll

We will return to the Art Box again and again as we go through the
system, illustrating how it can replicate in the Geometron system and
how it fits into all the parts: symbols in ``symbols'' for the shapes
and net, the set in ontology, maps, feeds, scrolls, scroll replicator,
role in magic as one of the Trash Robot avatar objects

The user page, basic operation of Geometron as it exists today, just
viewing now editing, but with pointers to editors, jump on and look at a
file RIGHT NOW, on one of my pages.

In organic media chapter we introduce geometron and say what we're going
to be doing in this book. What we are doing is providing a
\emph{specification} and an \emph{example} of an instance of that spec.
We will use the Pi, and remote hosted pages. But expect this spec to be
used by many coders on many platforms in many ways if successful.
