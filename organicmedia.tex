

\section{Organic Media}\label{organic-media}

In order to build a world of abundance from waste streams of the old world we need to create a new type of media which self-replicates.  We need the media to replicate, the means to produce the media to replicate, and for the media to function as the replication medium for other things of all kinds.  In order to build this new type of media we will in this chapter first explore exactly what self-replicating media is and how it works.  We refer to self-replicating media as  being ``organic'' media.


Some laws our media will follow in its construction are:

\begin{itemize}
\item
  everything replicates
\item
  everything evolves
\item
  everything dies
\item
  no money
\item
  no mining
\item
  no property
\end{itemize}

Organic media exist to replicate that which we desire to replicate.  Its purpose is to facility replication of things.  It contains both information on how to replicate things and information required to inspire people to \emph{desire} to carry out replication.  We must give people a reason to replicate as well as a means to replicate.  Thus one form of Organic Media is instructions on how to build something you can sell.  To maximize replication we are less concerned with how much money someone can make via replication than we are with how little they have to spend to initiate replication.  We want to minimize the cost of each item a person might buy to sell, the minimum number required to be purchased, and also minimize the amount of labor and skill required to make the thing.  The ArtBox is an example: each box contains the content to make another box.  Each box is awesome and can be sold.  The ArtBox itself is Organic media.  a bunch of SRS gets moved here.


path of replication: how to install and how to teach others to install,
how to replicate the code, replicate specific files, how it's all
structured, editing the code itself

the structure of sign pointing to domain which points to terminal which
is near a place which is labelled by domain(this needs to get explained
early since it will be the referenced in everything, as it's basic to
the structure of the System)

Editing: how files work, how they don't work. No databases. no private
data. No file that cannot be edited.

Deletion: cheap sd cards can be wiped any time. All files can be deleted
or written over by anyone at any time. Domains are chosen to be of no
real value, are disposed of as soon as burned, never sold. Evaluate
entropy of domain names to prove we can always have zero value of names
even as our network grows exponentially. ``Permanent'' deletion less
critical in a world without private or personal data.

Cybersecurity: there is none. Security is for private property, and
there simply is none on this network. This network must always be
disconnected from the property-based Internet or private systems. One
way paths of information for protection.

draw from the SRS paper and the existing organic media paper(but the
formal SRS idea goes in ontology chapter)

File types independent of technology: symbols, feeds, maps, scrolls.
These can be described without any specific implementation of
\emph{either} software \emph{or} hardware. Specific implementations will
be dealt with in their own sections.

How files are structured, overview of file types, what they do. All of
this is as a generalized specification which users can use to rewrite
from scratch quickly.

Example of physical organic media: The Art Box, replicator card,
replicator shape, object itself, scroll

We will return to the Art Box again and again as we go through the
system, illustrating how it can replicate in the Geometron system and
how it fits into all the parts: symbols in ``symbols'' for the shapes
and net, the set in ontology, maps, feeds, scrolls, scroll replicator,
role in magic as one of the Trash Robot avatar objects

The user page, basic operation of Geometron as it exists today, just
viewing now editing, but with pointers to editors, jump on and look at a
file RIGHT NOW, on one of my pages.

In organic media chapter we introduce geometron and say what we're going
to be doing in this book. What we are doing is providing a
\emph{specification} and an \emph{example} of an instance of that spec.
We will use the Pi, and remote hosted pages. But expect this spec to be
used by many coders on many platforms in many ways if successful.

This is the chapter which defines Geometron, states the goal here.  We want an information network which facilitates the free replication of technology built from trash.  To do that, we will need text documents, hypertext graphical documents(like slides), feeds of information elements, and generalized symbols and symbolic languages which can be used to create vector graphics and control machines for automation.  Creating this language and a hardware platform to transmit it, which replicates itself will be a seed of a new economic system.  In order for this to self-replicate we need the media to create value for its users.  The next section describes the physical network which our system runs on.   

In this work, we describe a system called Trash Robot and a language and IT platform called Geometron which together can serve as a platform for a post-scarcity economy without money.  We will describe how the symbolic language can be used to create automation technology from scavenged trash components, and demonstrate with a practical machine which itself is used to fabricate materials which can then be sold.  

\subsection{Trash Robot}

Trash Robot is a form of organic media which will be transmitted via the Street Network documented in the next section and the Geometron software platform described below and in the entirity of this book.  Trash Robot is a combination of fabrication technology built from trash(the robot itself), self-replicating media consisting of clay coin-like tokens printed with the robot which can be used to stamp out more of themselves by repeatedly molding and baking batches of clay, and some other elements which are also designed to easily replicate.  Trash Robot replication is stimulated by the fact that it can make money.  People will buy robots, and the clay tokens can have any symbol on them, and can be made into jewelry and coins which can also be sold for money.  As long as there is money to be made from replication we should expect replication to take place freely.  A robot can be built for under 50 dollars in parts and easily sold for well over 100 dollars, probably quite a bit more than that.  And the products of the robot can be sold on the street for dollars but cost pennies to make.  

As with all media in our system, the Trash Robot belongs to no one when it is made.  We can sell it and then it belongs to someone but they can also pass it along for free, share it, copy it, and sell it.  The fact that it is not property when it's initially created will tend to make it move freely around and replicate.



\subsection{Geometron}

This work loops back in on itself over and over, as does anything that self-replicates.  Just as every living cell in our bodies carries a whole vast self-contained and self-replicating system with all our genetic information, so this book must return again and again to describing itself it it is going to work to replicate the system.  To that end this final section in the discussion of organic media describes just what Geometron is, what it is for, and why it is organic media.

Geometron is an information technology system on which we will build a new civilization beyond the artificial scarcity of the mining and consumption based civilization in which we all presently live.  This system will be built on technology which is readily available all over the globe for very little money combined with material from waste streams like cardboard and plastic trash. As we grow, however, we will replace every component of the system which currently comes from a factory and mine with one which comes from a waste stream until there are no longer any inputs from the consumption-based civilization.  The last chapter of this work will discuss the technological roadmap to get to that point using a hybrid technology of found electronics components from waste streams with components fabricated directly using fabrication technology which is itself built from waste components and which uses either organic materials or waste streams as its feedstock.  Trash Robot serves as the seed of this, as it is a system built partly from waste(mechanical parts from old DVD or CD drives) which can be used for a wide variety of fabrication tasks.  The Machine Control chapter will discuss how the Geometron software can be used for building practical fabrication and automation technology at all scales, from the scale of 100 meters or even 1000s of meters for agriculture down to the sub-micron scale for nano or micro electronic circuit fabrication, along with all scales in between.  

In order to build all this, we have to start with a social media platform built entirely around the tasks we wish to carry out.  These are first and foremost building community of like-minded people who have the desire to build this better system. Then it is about as a group building things which are of direct utility to both the people in the network and other people we interact with.  If we can provide tangible value, both in terms of goods and services, our network will naturally replicate, as people can both get money for growing the network and get direct non-monetary benefits(with new and interesting products they can use and share).  We also need the network to not just blindly grow but to have an active research and development effort in which we as a community build out the technologies along the roadmap to a closed cycle organic civilization.  This means our platform needs to have all the basic tools we need to do that research: automation, measurement, control, graphical communication, technical communication of all kinds, code development, fabrication, numerical processing and free open replacements for all the basic tools used for modern technology development.  

Geometron is a seed for such a network.  I have created this system to run in web browsers, which run on billions and billions of devices, in dozens of web browsers using the standards which are univeral today on the World Wide Web. Documents are created by users on apps which allow them to create, edit, copy and destroy everything with no ownership of any files.  Servers are small portable computers carried in a cloth shoulder bag which can be carried on the
 person of the Operators of the network. These are fully functional web servers, but are intended for use on local private networks in a physical place with a shared wifi network.

\subsection{The Book of Geometron}

This book itself is organic media. It is intended to teach its contents to a reader(or rather a small subset of readers) to the level where they can then teach another.  This should enable them to re-write future improved versions.  In this section I describe how the book was put together, where the files are stored, how to edit them and use the \LaTeX document preparation system to make the files required to produce a finished book.  This means you also need to know what is required to get the physical book printed at an on demand printer, get all the metadata required for publication and distribution, and sell your version in retailers both large and small and online and off.  Thus even the book is fully decentralized in principle: if it costs you nothing to set it up to sell, you can sell only a half dozen copies and it will be a net positive, and then if the next person does this it will also be positive and so on.

If the book is decentralized in this way of distribution it has many advantages.  If the book turns out to be disruptive enough that people try to use lawsuits to shut it down or harass an author, but there are 10's of thousands of new authors popping up all the time, it will be impossible to shut down. As some versions turn out to be dangerous or illegal, other versions can immediately be published with omit the offending content. Also, many editions will mean some will get much better than this initial manuscript.  Decentralization means that as the manuscript finds its way into communities that speak different languages, the translation can happen without any centralized effort. So for instance if someone translates from the original English into say French, and then it spreads around in areas bilingual with French and some other language like Swahili it can go directly from the French to the Swahili without any involvement of the initial English speaking writers.  By avoiding copyright, these improved and translated versions can then get translated back into English and sold yet again under yet anotehr edition.  Having editions be unique can create a market for unusual editions, further pumping money into the system and stimulating further development of the book.  I would rather see 10,000 people make 100 dollars each selling their own editions of this book to just their friends than see me as the initial author make 1 million dollars on 500,000 copies of one edition.

It is not my intent in the long run to make money on Geometron. It is my intent to create a network which allows us to live without money by directly bartering what we need to survive(food, a place to sleep and work, medicine, transport) without use of money or any production in the old consumer economy.

All editions are published with a public domain license for everything but the final pdf.  The final pdf is published under the minimal copyright required for an author to create the needed publication metadata to get distribution outside of the on demand press used for printing.

Each chapter is a .tex file, using standard \LaTeX.

This work must replicate itself completely.  We show here how to edit each chapter, publish them to a public Github repository with detailed instructions for further replication, compile the document to a .pdf in book format, and self-publish the fully compiled book on Lulu Press. We then guide the reader to follow the instructions on Lulu to get all the needed copyright metadata for official distribution through normal publication channels.  We then describe how to order just a few copies, sell them along with other parts of the system here at a markup, and use the profits to buy more print copies of their own book to place in bookstores and libraries as a fully guerilla activity with no official sanction.  This is a little twist on the methodology of Abbie Hoffman's ``Steal This Book.''  In Steel This Book, book sellers had to buy the book, which readers inevitably stole, cutting into book store profits.  We use guerilla production methods to distribute it into bookstores without them spending money.  They are faced with a choice: go along with our program and take free money from customers for the book or make trouble for us, throw the book out, and loose what is for them totally free money.  If they sell the books for a profit, it benefits our network, because it spreads our ideas but also because it creates a value stream in the existing economy based on what we create, which gives us power in that network.   Bookstores want money.  But we want network centrality and the ability to control how information flows in a network, and free distribution shifts the power to us.  This is why the self-replication of the book is so important.  If you want to make your own spin on this book and make it more of a best seller, you do it, but if you leave it open, you hope someone else does it again, and that it keeps getting better as it replicates.  